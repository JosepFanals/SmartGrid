% 1.0 introduce smart grids and how they transform the system
% 1.1 literature review, cite
% 1.2 goals and scope of this work
% 1.3 how the project distributes the work and main contributions

In modern life, electricity has become a necessity and everyone, households like industries, has to be provided whenever needed. The challenge is that electricity cannot be stored and demand fluctuates. In order to do so, since the early 20th century, the electricity grid from producer to consumer has rapidly increased, building a great network. Nevertheless, even if now the grid is able to provide anyone with electricity, in today's context it is not enough anymore. The growing demand and the climate change crisis pushes toward a more efficient use of energy and greener production. The actual grid, mainly being built a century ago when renewable energy was not in the topic, is long as fossil fuel power plants need to be relatively far from cities and villages and weak in numerical control. This leads to a lot of losses in the transmission and distribution lines, up to 10\% \cite{councileu}. As a response to those issues, the smart grid concept emerged. Its definition varies but overall its aim is to improve and modernize the actual grid in order to improve its efficiency and reduce greenhouse effect in a cost effective way \cite{ahat2013smart}. For doing so, smart grids can use faster data communication thanks to numerical meters, implementation of proximate electricity sources such as household PV, new electricity market for prosumers, wiser selection of high voltage lines, etc. The smart grid concept is relatively new and because testing it in real life is expensive it is first modeled and optimized before being implemented. Moreover, to optimize the conventional grid, this latest has to be well understood.

Many previous works have been conducted, mainly starting from 2008 \cite{vakulenko2021systematic}. They mostly focus on the engineering, energy and computer science aspects. Reports in the field of environmental science are about 7\% of all the publications and those relating to economical aspects, although their number is increasing, they remain very low. Studies can be separated in two categories, those related to specific elements of smart grids and those optimizing the whole. In the first case, every element of smart grids is discussed: the power generation with renewable energy sources, transmission and distribution (bidirectional flow), storage, end user, microgrid, market, meters and communications. New technologies can be implemented in order to control and heal the system in an automated way. For example, Automatic Voltage Regulation (AVR) keeps the voltage within the limits, Energy Management System (EMS) ensures stability of every operating point, Automatic Generation Control (AGC) performs optimal load distribution among the generating units, Advanced Metering Infrastructure (AMI) regroups the meters devices, Meter Data Management (MDM) allows fast bidirectional communication of the measures and helps the decision-making, and others like Distribution Management System (DMS), Geographical Information System (GIS), Outage Management System (OMS), Wide Area Management System (WAMS), and Demand Side Management (DSM) \cite{alotaibi2020comprehensive}. 

On the other hand, research has been done in order to optimize all those elements \cite{gao2021review}. Most of the time, optimization is simplified by focusing on a single objective at a time. The main objectives are energy performance, economical performance or environmental protection. Nevertheless, multi-objectives optimizations are more likely to be accurate. The two main multi-objective optimization models are the weighting factor method and the pareto model, based on the necessary and sufficient conditions logic. Then the decision variables and the constraints are to be added. The decision variables are values that decision-makers have to decide and to do so several multi-criteria methods have been established. For example, Gu et al. wrote a couple of widely accepted evaluation methods and indicators for CCHP microgrid planning \cite{gao2021review}. Within the constraints, we can cite the power balance, generation capacity, transmission capacity between a microgrid and a large grid, location of the microgrid energy storage system,... The minimum operational cost is also a strong constraint. Different algorithms regrouping the objectives, the decision variables and the constraints have been written. The most frequently used is the genetic algorithm, based on Darwin’s selection principle, it selects within all the solutions the best ones each generation but others such as the simulated annealing method, the particle swarm optimization and others are also frequently used.

In this work, a simplified conventional network is considered. This initial network, composed of 4 loads, 1 nuclear plant, a dismantled gas plant and an interconnection with an external electricity grid, does not answer the requirement of a basic grid, i.e. providing electricity at all time. The goal is thus to improve this network. The first stage is to analyze the existing grid and point out the weaknesses in order to suggest solutions. The second stage consists of the improvement of the system thanks to the addition of lines and voltage increase. The third phase improves the network by integrating renewable energy sources. The fourth phase rehabilitates the dismantled gas plant and reduces energy dependence by integrating storage and control. Finally, the Smart Grid Architecture Model (SGAM) for the control is designed. For every improvement, a contingency analysis is made, i.e. the network reaction in case one element fails. 

