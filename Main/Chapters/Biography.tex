\textbf{Víctor Escala García} received the B.S. degree in industrial engineering from ETSEIB (UPC), Barcelona, Spain, in 2020. After that, he is currently finishing the Energy Engineering Master focused on renewable energy. Meanwhile, he has been on an internship with RWE Renewables working on the performance analysis team. He is especially motivated about renewable energy, data science and artificial intelligence. 

\textbf{Josep Fanals i Batllori} received the B.S. degree in electrical engineering from Universitat de Girona (UdG), Girona, Spain, in 2020. In 2020, he developed the first open-source version of the holomorphic embedding load-flow method, which is currently available in GridCal. He is now pursuing the MSc in energy engineering, electrical specialty. Since 2021 he has been with the Centre d'Innovació Tecnològica en Convertidors Estàtics i Accionaments (CITCEA-UPC), where he has been developing tools for the analysis of power networks. His current interests include numerical methods for the resolution of modern power grids dominated by power electronics. 

\textbf{Pol Heredia Julbe} received the B.S. degree in energy engineering from EEBE (UPC), Barcelona, Spain, in 2021. That same year, he started the Electric Power Systems and Drives Master, which is focused on power systems, the introduction of renewables in the grid and the modernization of power networks. Since 2020 he is working as a project engineer at IREC (Institut de Recerca en Energia de Catalunya), participating in projects related to network stability and the effects of distributed energy resources (DER) on the network.

\textbf{Roger Izquierdo Toro} received the B.S. degree in industrial engineering from ETSEIB (UPC), Barcelona, Spain, in 2020. His bachelor's final project was about the control and simulation of power converters operating in grid-forming mode. Since 2020 he is studying the master's degree in Energy Engineering from UPC. His work experience includes the position of a technical officer from 2019 to 2020 in Citelum Ibérica S.A., which is the company in charge of the public lighting infrastructure in Barcelona. He is currently working as an engineer in Agència de l'Habitatge de Catalunya, which is an institution of the Catalan administration that carries out the restoration and maintenance of social housing. 

\textbf{Palina Nicolas} graduated from the Université Libre de Bruxelles in Belgium with a B.S. in bioengineering in 2020. Concerned about environment and energy being one of the most polluting sectors yet essential and source of inequality, she decided to focus on renewable energy, pursuing with a master degree at the ETSEIB (UPC). 

