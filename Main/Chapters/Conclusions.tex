This project has covered the development of a grid, from its traditional structure to a system that integrates renewables and is more in line with the objectives of smart grids. Overall, this work has been splitted in four phases.

The first phase presents the basic system as such. It has been shown that there are two generation sources and four loads. While generation buses are able to set a voltage level, and hence they operate at acceptable levels independently of the hour of the day, loads suffer variations in voltages. This problem is especially acute the further they are from the generation sources. Voltages take values below 0.9~p.u. during peak hours, and although lines are not overloaded, they almost reach the limits. 

Phase 2 solves the aforementioned issues by installing more lines. Additional lines increase the reliability of the system --- therefore they may allow to operate under the $N-1$ criteria --- but they suppose an increase in cost. Contingency analysis have been performed, where hundreds of combinations have been tested. It was found that up to four extra lines were required to meet the $N-1$ criteria. Besides, raising the voltage level of the whole grid is another option that has been considered. Nonetheless, it becomes more expensive than installing additional lines.

On the other hand, phase 3 integrates renewables into the system. In more detail, a wind power plant and a photovoltaics power plant are included. They are next to the loads, which helps a bit at raising the voltages, reducing the power losses and the loadings of the lines. However, their power is considerably lower than the nominal power of the nuclear power. Although they do not exert a huge impact, renewables allow to meet the $N-1$ criteria while having to install less additional lines.

Phase 4 considers the installation of a storage system and a traditional power plant. The capacity of the storage system is just a fraction of the nominal power of the generation sources. Consequently, it had a very small effect on the results. The dismantled power plant becomes a solid selection from the point of view of operation of the system. It is relatively close to the loads, and permits having to add even less power lines to meet all the requirements. Its downside is found in the associated environmental impact. Several cases have been analyzed depending on the fuel it used. 

Additionally, the SGAM model has been employed to define a high level use case. In particular, it is focused on contingency analysis. First, a description of the use case, with the objectives and the relation to other use cases has been presented. Then, the component, business and function layers have been filled according to the specifications. Even though primary use cases are not a central part of this work, they have been included to show the full picture of the contingency analysis.

It is also worth mentioning that the simulations have required writting code to compute the power flow of multiple scenarios. The chosen programming language has been Python, due to its flexibility. Mainly, the functions have relied on Pandapower as the main package to call in order to solve the power flow. In this sense, it has been attempted to produce robust code based on object-oriented programming principles. The code is presented in the annex, and it can also be found in GitHub \cite{repoo}. 
