This chapter covers the inclusion of renewables into the initial grid. It has been verified that the system, as it is configured in the beginning, lacks resiliency. One of its major issues is the presence of only two generation units placed far from the loads. This causes some lines to exceed their loading capacity, while voltages' absolute value may be well below 0.9 p.u.. Adding new lines is a possibility, or rather, a must in order to meet the $N-1$ criteria. However, it is hypothetized that the performance of the system can still improve with the installation of a wind farm and a PV power plant.

To verify the above-presented hypothesis, the chapter first presents a general overview of the system, along with the sizing of the renewable power plants. Then, the optimal configuration found in the previous chapter is analyzed with the inclusion of renewables to verify the improvement. Finally, a contingency analysis takes place to assess the $N-1$ criteria. 

\section{Overview}
Figure \ref{fig:netrene} presents the general scheme of the system with the solar and wind plants connected through two parallel lines to nearby buses. Again, new lines that could become suitable for ensuring the robustness of the system are drawn. 

\begin{figure}[!htb]\centering

  \begin{circuitikz}[/tikz/circuitikz/bipoles/length=1cm, line width=0.8pt]

    % grid
    \draw[gray!50!white, line width=0.5pt] (0.0,0.0) to [short] (8.0,0.0);
    \draw[gray!50!white, line width=0.5pt] (0.0,1.6) to [short] (8.0,1.6);
    \draw[gray!50!white, line width=0.5pt] (0.0,3.2) to [short] (8.0,3.2);
    \draw[gray!50!white, line width=0.5pt] (0.0,4.8) to [short] (8.0,4.8);
    \draw[gray!50!white, line width=0.5pt] (0.0,6.4) to [short] (8.0,6.4);
    \draw[gray!50!white, line width=0.5pt] (0.0,8.0) to [short] (8.0,8.0);

    \draw[gray!50!white, line width=0.5pt] (0.0,0.0) to [short] (0.0,8.0);
    \draw[gray!50!white, line width=0.5pt] (1.6,0.0) to [short] (1.6,8.0);
    \draw[gray!50!white, line width=0.5pt] (3.2,0.0) to [short] (3.2,8.0);
    \draw[gray!50!white, line width=0.5pt] (4.8,0.0) to [short] (4.8,8.0);
    \draw[gray!50!white, line width=0.5pt] (6.4,0.0) to [short] (6.4,8.0);
    \draw[gray!50!white, line width=0.5pt] (8.0,0.0) to [short] (8.0,8.0);


    % generators
    \draw (3.2,-1.8) to [sV, fill=magenta!70!cyan] (3.2,-1.2);
    \draw [short] (3.2,-1.2) to [short] (3.2,-1.0);
    \draw (-1.8,1.6) to [sV, fill=green!50!white] (-1.2,1.6);
    \draw [short] (-1.2,1.6) to [short] (-1.0,1.6);
    \draw (9.2,1.6) to [sV, fill=cyan!50!white] (9.8,1.6);
    \draw [short] (9.2,1.6) to [short] (9.0,1.6);
    \draw (4.8,3.6) to [sV, fill=black!60!white] (4.8,3.0);
    \draw [short] (4.8,3.6) to [short] (4.8,3.8);
    \draw (8.4,4.8) to [sV, fill=yellow!60!white] (9.0,4.8);
    \draw [short] (8.0,4.8) to [short] (8.4,4.8);
    \draw (1.6,9.2) to [sV, fill=orange!60!white] (1.6,9.8);
    \draw [short] (1.6,9.0) to [short] (1.6,9.2);


    % large buses
    \draw[line width=2.5pt] (2.7,0) to [short] (3.7,0);
    \draw[line width=2.5pt] (0.0,1.1) to [short] (0,2.1);
    \draw[line width=2.5pt] (8.0,1.1) to [short] (8,2.1);
    \draw[line width=2.5pt] (1.1,8) to [short] (2.1,8);
    \draw[line width=2.5pt] (2.7,3.2) to [short] (3.7,3.2);
    \draw[line width=2.5pt] (2.7,4.8) to [short] (3.7,4.8);
    \draw[line width=2.5pt] (4.3,6.4) to [short] (5.3,6.4);
    \draw[line width=2.5pt] (0.0,5.9) to [short] (-0.0,6.9);
    \draw[line width=2.5pt] (8,4.3) to [short] (8,5.3);
    \draw[line width=2.5pt] (4.3,4.8) to [short] (5.3,4.8);

    % small buses
    \draw[line width=2.5pt] (2.9,-1) to [short] (3.5,-1);
    \draw[line width=2.5pt] (-1.0,1.3) to [short] (-1,1.9);
    \draw[line width=2.5pt] (9,1.3) to [short] (9,1.9);
    \draw[line width=2.5pt] (4.5,3.8) to [short] (5.1,3.8);
    \draw[line width=2.5pt] (1.3,9.0) to [short] (1.9,9.0);

    \draw[line width=2.5pt] (1.75,3.2) to [short] (1.75,2.6);
    \draw[line width=2.5pt] (1.75,4.8) to [short] (1.75,4.2);
    \draw[line width=2.5pt] (-0.95,6.7) to [short] (-0.95,6.1);
    \draw[line width=2.5pt] (4.5,7.35) to [short] (5.1,7.35);

    % trafos
    \draw (3.2,-1.0) to [voosource] (3.2,0.0);
    \draw (-1,1.6) to [voosource] (0,1.6);
    \draw (8,1.6) to [voosource] (9,1.6);
    \draw (1.6,8) to [voosource] (1.6,9);
    \draw (4.8,4.8) to [voosource] (4.8,3.8);

    \draw (-1,6.4) to [voosource] (0,6.4);
    \draw (4.8,6.4) to [voosource] (4.8,7.4);
    \draw (2.7,4.5) to [voosource] (1.7,4.5);
    \draw (2.7,2.9) to [voosource] (1.7,2.9);

    % loads
    \draw[-{Triangle[length=5mm, width=2mm]}, draw=blue!60!white, fill=blue!60!white] (-1,6.4) -- (-1.6,6.4);
    \draw[-{Triangle[length=5mm, width=2mm]}, draw=blue!60!white, fill=blue!60!white] (4.8,7.4) -- (4.8,8.0);
    \draw[-{Triangle[length=5mm, width=2mm]}, draw=red!60!white, fill=red!60!white] (1.7,4.5) -- (1.1,4.5);
    \draw[-{Triangle[length=5mm, width=2mm]}, draw=blue!60!white, fill=blue!60!white] (1.7,2.9) -- (1.1,2.9);

    \draw (3.0,4.8) to [short] (3.0,4.5);
    \draw (3.0,4.5) to [short] (2.7,4.5);

    \draw (2.75,3.2) to [short] (2.75,2.9);
    \draw (2.75,2.9) to [short] (2.7,2.9);

    % lines
    \draw (3.15,0) to [short] (3.15,3.2);
    \draw (3.25,0) to [short] (3.25,3.2);
    \draw (3.15,3.2) to [short] (3.15,4.8);
    \draw (3.25,3.2) to [short] (3.25,4.8);
    \draw (4.6,4.8) to [short] (4.6,5.1);
    \draw (4.6,5.1) to [short] (3.4,5.1);
    \draw (3.4,5.1) to [short] (3.4,4.8);
    \draw (3.15,4.8) to [short] (3.15,5.1);
    \draw (3.25,4.8) to [short] (3.25,5.1);
    \draw (3.15,5.1) to [short] (4.55,6.1);
    \draw (3.25,5.1) to [short] (4.65,6.1);
    \draw (4.55,6.1) to [short] (4.55,6.4);
    \draw (4.65,6.1) to [short] (4.65,6.4);
    \draw (5.0,6.4) to [short] (5.0,6.1);
    \draw (5.0,6.1) to [short] (7.7,5.1);
    \draw (7.7,5.1) to [short] (8.0,5.1);
    \draw (0,6.4) to [short] (0.3,6.4);
    \draw (0.3,6.4) to [short] (3.0,5.1);
    \draw (3.0,5.1) to [short] (3.0,4.8);


    % proposed new lines
    \draw[dashed, draw=red] (8, 4.4) to [short] (3.5,4.4);
    \draw[dashed, draw=red] (3.5,4.4) to [short] (3.5,4.8);
    % \draw[dashed, draw=red] (8,4.4) to [short] (3.5,3.2);
    \draw[dashed, draw=red] (4.5,6.4) to [short] (0,6.4);
    \draw[dashed, draw=red] (3,0) to [short] (3,4.8);
    \draw[dashed, draw=red] (3,3.2) to [short] (0,6.2);
    \draw[dashed, draw=red] (3,0) to [short] (0,6.2);

    % \draw[dashed, draw=red] (8,5.0) to [short] (4.8,5.0);
    % \draw[dashed, draw=red] (4.8,5.0) to [short] (4.8,4.8);
    \draw[dashed, draw=red] (8,4.4) to [short] (3.3,0);

    \draw[draw=red] (6.0,4.25) to [short] (6.3,4.55);
    \draw[draw=red] (6.2,4.25) to [short] (6.5,4.55);

    % \draw[draw=red] (6,4.85) to [short] (6.3,5.15);
    % \draw[draw=red] (6.2,4.85) to [short] (6.5,5.15);

    % \draw[draw=red] (6,4.25) to [short] (6.3,4.55);
    % \draw[draw=red] (6.2,4.25) to [short] (6.5,4.55);

    \draw[draw=red] (2,6.25) to [short] (2.3,6.55);
    \draw[draw=red] (2.2,6.25) to [short] (2.5,6.55);

    \draw[draw=red] (2.85,1.5) to [short] (3.15,1.8);
    \draw[draw=red] (2.85,1.7) to [short] (3.15,2.0);

    \draw[draw=red] (2.3,3.6) to [short] (2.6,3.9);
    \draw[draw=red] (2.2,3.7) to [short] (2.5,4.0);

    \draw[draw=red] (5.3,1.6) to [short] (5.0,1.9);
    \draw[draw=red] (5.4,1.7) to [short] (5.1,2.0);

    % lines for rene and solid interconnection
    \draw (0,6.7) to [short] (0.3,6.7);
    \draw (0,6.8) to [short] (0.3,6.8);
    \draw (0.3,6.7) to [short] (1.65,7.7);
    \draw (0.3,6.8) to [short] (1.55,7.7);
    \draw (1.65,7.7) to [short] (1.65,8.0);
    \draw (1.55,7.7) to [short] (1.55,8.0);

    \draw (0,1.55) to [short] (2.85,1.55);
    \draw (0,1.65) to [short] (2.75,1.65);
    \draw (2.85,1.55) to [short] (2.85,3.2);
    \draw (2.75,1.65) to [short] (2.75,3.2);

    \draw (8,4.9) to [short] (4.9, 4.9);
    \draw (8,5.0) to [short] (4.8, 5.0);
    \draw (4.9,4.9) to [short] (4.9,4.8);
    \draw (4.8,5.0) to [short] (4.8,4.8);

    \draw (8.0,4.4) to [short] (7.7,4.4);
    \draw (7.7,4.4) to [short] (3.5,3.5);
    \draw (3.5,3.5) to [short] (3.5,3.2);

    % legend
    \draw[-{Triangle[length=5mm, width=2mm]}, draw=blue!60!white, fill=blue!60!white] (3.0,8.4) -- (4.0,8.4);
    \draw[-{Triangle[length=5mm, width=2mm]}, draw=red!60!white, fill=red!60!white] (3.0,9.1) -- (4.0,9.1);
    \draw (3.2,9.8) to [sV, fill=magenta!70!cyan] (3.8,9.8);

    \draw (6.0,9.8) to [sV, fill=black!60!white] (6.6,9.8);
    \draw (6.0,9.1) to [sV, fill=yellow!60!white] (6.6,9.1);
    \draw (6.0,8.4) to [sV, fill=green!50!white] (6.6,8.4);

    \draw (9.0,9.8) to [sV, fill=orange!60!white] (9.6,9.8);
    \draw (9.0,9.1) to [sV, fill=cyan!50!white] (9.6,9.1);
    \draw[dashed, draw=red] (9.0,8.4) to [short] (9.5,8.4);

    \node at (4.73, 9.8) {\footnotesize Nuclear};
    \node at (4.66, 9.1) {\footnotesize Load I};
    \node at (4.72, 8.4) {\footnotesize Load II};

    \node at (7.64, 9.8) {\footnotesize Dismantled};
    \node at (7.49, 9.1) {\footnotesize Intercon.};
    \node at (7.28, 8.4) {\footnotesize Wind};

    \node at (10.25, 9.8) {\footnotesize Solar};
    \node at (10.40, 9.1) {\footnotesize Storage};
    \node at (10.45, 8.4) {\footnotesize New line};

    \draw[gray!50!white, line width=0.5pt] (8.5,8.0) to [short] (10.1,8.0);
    \draw[gray!50!white, line width=0.5pt] (8.5,6.4) to [short] (10.1,6.4);
    \draw[gray!50!white, line width=0.5pt] (8.5,6.4) to [short] (8.5,8.0);
    \draw[gray!50!white, line width=0.5pt] (10.1,6.4) to [short] (10.1,8.0);

    \node at (9.30, 6.2) {\footnotesize 50 km};
    \node[rotate=90] at (10.3, 7.2) {\footnotesize 50 km};

    \draw [fill=gray, opacity=0.2, line width=0.01pt] (2.95,10.15) rectangle (11.5,8.05);
    \draw [fill=gray, opacity=0.2, line width=0.01pt] (11.5,8.05) rectangle (8.05,6.05);

    % buses nodes and labels
    \node at (3.6, 0.2) {8};
    \node at (3.6, -1.0) {1};
    \node at (3.6, 4.6) {9};
    \node at (1.75, 5.0) {2};
    \node at (3.6, 3.0) {10};
    \node at (1.75, 3.45) {3};
    \node at (4.4, 6.6) {11};
    \node at (4.4, 7.6) {4};
    \node at (0.0, 7.1) {12};
    \node at (-1.2, 6.7) {5};
    \node at (5.2, 4.6) {13};
    \node at (5.3, 3.7) {7};
    \node at (7.8, 4.6) {6}; 

    \node at (0.9, 8.0) {15};
    \node at (1.1, 9.0) {14};

    \node at (0.0, 0.9) {16};
    \node at (-1.0, 1.1) {17};
    % \node at (1.3, 8.0) {14};

  \end{circuitikz}

  \caption{Overview of the network with renewables and the potential addition of lines}
  \label{fig:netrene}
\end{figure}
The optimal configuration with no renewables, as shown in Table \ref{tab:top10_1}, requires the installation of four additional lines: 6-13, 6-10, 8-9, 10-12. The first two are pretty much found in every other topology, as they represent a link with the interconnection. Since the interconnection can theoretically provide any power, it is a good idea to increase the number of associated lines to it. Consequently, lines 6-13 and 6-10 are not considered as potentially attractive new lines, but as already connected lines.



\section{Renewable integration}
In this project, the integration of renewables takes place in the form of solar and wind. The characteristics of each one are particularly detailed. The goal is to characterize their hourly variation of power in relation to the availability of the natural resources. Despite the partial unpredictability that renewables suffer, a representative day in terms of solar irradiance and wind speed has been selected. 

\subsection{Solar PV}
First, the PV panels cover an area of 60~ha. This, and some other relevant information is captured in Table \ref{tab:pv}. 

% - Put some equations to calculate the power from solar or wind

\begin{table}[!htb]\centering
  \begin{tabular}{ccc}
    \hline
    \textbf{Magnitude} & \textbf{Value} & \textbf{Units}\\
    \hline
    Location & (40.8N; 0.48E) & -\\
    Total area & 60 & ha \\
    System efficiency & 20.7 & \% \\
    % Panel model & 405W Deep Blue 3.0 JA Solar Mono & - \\
    Cell type & Monocrystalline & - \\
    Panel peak power & 405 & Wp \\
    Panel area & 1.95 & m$^2$ \\
    Open circuit voltage & 37.23 & V \\
    Short circuit current & 13.87 & A \\
    \hline
  \end{tabular}
  \caption{Technical characteristics of the PV plant with its panels 405W Deep Blue 3.0 JA Solar Mono \cite{autosolar}}
  \label{tab:pv}
\end{table}
The location has a certain importance to determine the incident irradiation, which is extracted from PVGIS database \cite{pvgis}. In reality, the solar power plant would be located in Tortosa, Catalunya, Spain, according to the given coordinates. The chosen PV panels are of monocrystalline type. Monocrystalline cells generally offer larger efficiencies than polycrystalline cells \cite{kumar2015comparative}, and although they become more expensive, no cost limitations have been set. 

Let $n_p$ denote the total number of panels, $G$ the irradiance, $\eta$ the efficiency and $S_p$ the area of a single panel. Then, the total generated power follows:
 \begin{equation}
   P_{PV} = n_p G \eta S_p.
   \label{eq:pvx}
\end{equation}
It has to be noted that the irradiance $G$ is varying along the day, and it generally has a peak around noon. The data have been selected for the 15th of February. There are a total of $n_p=63692$~panels as it has been assumed that the 60~ha surface is the useful one. Figure \ref{fig:pvpx} shows the hourly irrandiance and power of the PV plant. Clearly, they are proportional, just like it has been already shown in Equation \ref{eq:pvx}.

\begin{figure}[!htb]\centering
\begin{tikzpicture}
    \begin{axis}[xlabel={Hour}, ylabel={$G$ (W/m$^2$)}, axis y line*=right, grid=both, grid style={line width=.1pt, draw=gray!10}, major grid style={line width=.2pt,draw=gray!50}, xtick distance = 2, ytick distance = 100, width=12cm, height=6cm, every plot/.append style={very thick}, xmin = 1, xmax = 24, ymin=0.0, ymax=900, very thick, grid=both, grid style={line width=.4pt, draw=gray!10}, major grid style={line width=.8pt,draw=gray!50}, legend style={at={(0.8,0.1)},anchor=south west}]
        \addplot[color=orange] table[col sep=comma, x=x, y=y] {Data/phase3/G.csv};
        \legend{$G$}
\end{axis}

    \begin{axis}[xlabel={Hour}, hide x axis, ylabel={$P$ (MW)}, axis y line*=left,  xtick distance = 2, ytick distance = 10, width=12cm, height=6cm, every plot/.append style={very thick}, xmin = 1, xmax = 24, ymin=0.0, ymax = 96, very thick, legend style={at={(0.8,0.25)},anchor=south west}]
        \addplot[color=red] table[col sep=comma, x=x, y=y] {Data/phase3/Ppv.csv};
        \legend{$P$}
\end{axis}

\end{tikzpicture}
\caption{Irradiance and power from the PV plant along a representative day. Data from \cite{pvgis}.}
    \label{fig:pvpx}
  \end{figure}
The peak in power takes place at hour 13, more or less as expected, and the curves follow a profile that resembles a Gaussian bell. The maximum power generated by the PV plant is 91.5~MW. Although this is a respectable amount, it is roughly one tenth of the load peak. As a consequence, results are not expected to change signficantly. Most likely we will observe slight improvements.

\subsection{Wind}
On the other hand, wind power is proposed as the second renewable alternative. The chosen wind turbines are the Siemens-Gamesa G132-5.0MW. They are rather voluminous turbines, with a rotor diameter of 132~m. Table \ref{tab:wind} shows their most notorious technical characteristics.

\begin{table}[!htb]\centering
  \begin{tabular}{ccc}
    \hline
    \textbf{Magnitude} & \textbf{Value} & \textbf{Units}\\
    \hline
    Location & (42.28N; 3.16E) & -\\
    Total area & 80 & ha \\
    Nominal power & 5.0 & MW\\
    Cut-in wind speed & 1.5 & m/s\\
    Rated wind speed & 13.0 & m/s\\
    Cut-out wind speed & 27.0 & m/s\\
    Swept area & 13685 & m$^2$\\
    \hline
  \end{tabular}
  \caption{Technical characteristics of the wind power plant with its turbines Gamesa G132-5.0MW \cite{gamesa}}
  \label{tab:wind}
\end{table}
The power provided by a wind turbine is given by:
\begin{equation}
  P_{wt} = \frac{1}{2}\rho A C_p v^3,
\end{equation}
where  $\rho$ is the air density of approximately 1.225~kg/m$^3$, $A$ stands for the swept area, $C_p$ is the power coefficient limited to $16/27$ (known as Betz's limit \cite{ragheb2014wind}), and $v$ represents the wind speed. If the power $P_{wt}$ exceeds the nominal power of the turbine, some power will be curtailed so as not to surpass the machine's ratings. In normal operation, the power coefficient is likely to take values around 0.4.

The available area to install the wind turbines is 80~ha. Nonetheless, wind turbines have to be generously separated one from the others. It has been decided that wind turbines should be separated by a minimum of 5 diameters. Similar separations are adopted in the literature \cite{bartl2012wake}. Thus, a total of two wind turbines are installed. The nominal power of the wind farm becomes 10~MW, which is significantly lower than PV. Despite that, this should not come as a surprise since wind power has a relatively low power density \cite{smil2015power}.  

With this, Figure \ref{fig:windx} shows the evolution of the wind speed and the total generated power by the wind farm in an orientative day. 

\begin{figure}[!htb]\centering
\begin{tikzpicture}
    \begin{axis}[xlabel={Hour}, ylabel={$v_w$ (m/s)}, axis y line*=right, grid=both, grid style={line width=.1pt, draw=gray!10}, major grid style={line width=.2pt,draw=gray!50}, xtick distance = 2, ytick distance = 1, width=12cm, height=6cm, every plot/.append style={very thick}, xmin = 1, xmax = 24, ymin=0.0, ymax=12, very thick, grid=both, grid style={line width=.4pt, draw=gray!10}, major grid style={line width=.8pt,draw=gray!50}, legend style={at={(0.8,0.1)},anchor=south west}]
        \addplot[color=orange] table[col sep=comma, x=x, y=y] {Data/phase3/vw.csv};
        \legend{$v_w$};
\end{axis}

    \begin{axis}[xlabel={Hour}, hide x axis, ylabel={$P$ (MW)}, axis y line*=left,  xtick distance = 2, ytick distance = 1, width=12cm, height=6cm, every plot/.append style={very thick}, xmin = 1, xmax = 24, ymin=0.0, ymax = 10, very thick, legend style={at={(0.8,0.25)},anchor=south west}]
        \addplot[color=red] table[col sep=comma, x=x, y=y] {Data/phase3/Pwind.csv};
        \legend{$P$};
\end{axis}
\end{tikzpicture}
\caption{Wind speed and output power from the wind farm. Data from \cite{nasa}.}
    \label{fig:windx}
  \end{figure}
Notice the cubic relationship between the wind speed and the power. Small increases in wind speed provoke large variations in the otuput power. 

\section{Base case analysis}
This base case analysis refers to the assessment of the influence the installation of renewables has on the results. It is compared to the system analyzed before with lines 6-10 and 6-13 permanently added. The goal is not to deduce if the grid meets the $N-1$ criteria (most likely some other lines would have to be added), but rather compare if the voltages, the loadings of the lines and the losses suffer a certain improvement.  

To perform the comparison, the most extreme values have been gathered. They are shown in Table \ref{tab:compare} for both cases (with and without renewables). 

\begin{table}[!htb]\centering
  \begin{tabular}{lrr}
    \hline
    \textbf{Attributes} & \textbf{Without renewables} & \textbf{With renewables}\\
    \hline
    $V_{min}$ (p.u.) & 0.962 & 0.968 \\
    $V_{max}$ (p.u.) & 1.050 & 1.050 \\
    Max. load (\%) & 41.65 & 40.76 \\
    Max. losses (MW) & 14.54 & 14.43 \\
    Correct operation? & Yes & Yes \\
    \hline
  \end{tabular}
  \caption{Main results to compare between the grid with and without renewables}
  \label{tab:compare}
\end{table}
The results suggest that generally speaking there is little difference in installing or not renewables. The inclusion of the wind farm and the PV power plant cause the minimum voltage to rise a bit, the maximum loading of the lines to decrease about 1\% (although without contingencies it is already low) and the maximum power losses experience a tiny reduction. In the two scenarios, the maximum voltage remains 1.05, which indicates there are no overvoltages. Since all magnitudes are inside the allowed limits, the system is operating correctly in both cases.

Adding renewables to the system comes at a cost. First, there has to be an investment of capital for deploying the wind turbines and the PV panels, apart from power electronics equipment and ancilliary components. Then, a couple of new power lines are added to interconnect them to the system. The slight improvement on the electrical magnitudes will most probably fail at justifying this investment. 

However, it is important to note that the major advantage of installing renewables is in its environmental impact. If the power from the interconnection is assumed to come from coal and gas power plants, then installing renewables could become a sensible option. This is especially true when taking into account the actual costs of carbon emissions. Currently at 81.65~\texteuro/tonne, they have increased by a factor of 3 during the last year \cite{carbon}. 




\section{Contingency analysis}
A different story would be the operation with contingencies. Here it is valuable to find out if the presence of distributed renewable generators can help at having to install less extra lines. Hence, the cost could be reduced. 

The contingency analysis has been performed similarly to the ones in Tables \ref{tab:top10_1}, \ref{tab:cost_trafos2}. We take a set of lines that can be potentially installed, and a set of initially powered lines. Contingencies are caused in this latter set to study which new lines have to be connected to meet the requirements. Then, the topologies that ensure the correct operation of the system in spite of which line fails, are selected and their associated costs are calculated. The top 10 most convenient configurations are presented in Table \ref{tab:top10_rene}.

\begin{table}[!htb]\centering
  \begin{tabular}{ccc}
    \hline
    \textbf{Identifier} & \textbf{New lines} & \textbf{Infraestructure cost (M\texteuro)}\\
    \hline
    24 & [8-9, 10-12] & 402.71 \\
    12 & [8-9, 9-6] & 406.21 \\
    46 & [11-12, 8-9] & 406.21 \\
    8 & [8-9, 8-12] & 409.54 \\
    0 & [11-12, 8-9, 10-12] & 463.84 \\
    4 & [8-9, 10-12, 9-6] & 463.84 \\
    37 & [11-12, 10-12, 9-6] & 463.84 \\
    63 & [8-9, 10-12, 8-12] & 467.18 \\
    20 & [11-12, 8-9, 9-6] & 467.34 \\
    38 & [11-12, 8-9, 8-12] & 470.67 \\
    \hline
  \end{tabular}
  \caption{Best configurations with the additional lines}
  \label{tab:top10_rene}
\end{table}
Again, the optimal configuration is the one where apart from lines 6-10 and 6-13, lines 8-9 and 10-12 are added. This coincides with the conclusion we reached in Table \ref{tab:top10_1}. Nevertheless, the costs are now higher due to the addition of two new lines to connect the wind and the solar power plants to the system as such. Introducing renewables is not enough to avoid installing the same lines.

In case lines 8-9 and 10-12 are added, the operational costs would be practically zero because we would be meeting the $N-1$ criteria. To summarize, installing renewables implies an additional cost for the system operator. With time, this could be compensated with the savings in carbon emissions. 

%TODO: could trafos fail and then the operational cost would not be null!??!?

