While the previous phase included renewables to mitigate the technical problems, another possibility has to do with installing storage systems, rehabilitating a dismantled plant, among others. Certainly, renewables have not been enough to solve the issue of requiring a stronger interconnection. The same number of added lines have been needed in order to meet the $N-1$ criteria. Hence, it is convenient to evaluate if other elements are capable of improving the energetic independence of the system under study, and at the same time, if it becomes economically appealing.

This chapter is structured in the following manner. First, we detail the characteristics of the storage unit, its operation mode and the expected impact on the system. Then, the dismantled plant is evaluated. Focus is placed on its potential environmental impact rather than on its technical aspects. Once these elements have been defined, results are extracted by running the simulation, including the contingency analysis.

Just as an overview, Figure \ref{fig:netstorage} displays the system with the new elements under consideration.

% take the figure of the last chapter and add storage lines

\begin{figure}[!htb]\centering

  \begin{circuitikz}[/tikz/circuitikz/bipoles/length=1cm, line width=0.8pt]

    % grid
    \draw[gray!50!white, line width=0.5pt] (0.0,0.0) to [short] (8.0,0.0);
    \draw[gray!50!white, line width=0.5pt] (0.0,1.6) to [short] (8.0,1.6);
    \draw[gray!50!white, line width=0.5pt] (0.0,3.2) to [short] (8.0,3.2);
    \draw[gray!50!white, line width=0.5pt] (0.0,4.8) to [short] (8.0,4.8);
    \draw[gray!50!white, line width=0.5pt] (0.0,6.4) to [short] (8.0,6.4);
    \draw[gray!50!white, line width=0.5pt] (0.0,8.0) to [short] (8.0,8.0);

    \draw[gray!50!white, line width=0.5pt] (0.0,0.0) to [short] (0.0,8.0);
    \draw[gray!50!white, line width=0.5pt] (1.6,0.0) to [short] (1.6,8.0);
    \draw[gray!50!white, line width=0.5pt] (3.2,0.0) to [short] (3.2,8.0);
    \draw[gray!50!white, line width=0.5pt] (4.8,0.0) to [short] (4.8,8.0);
    \draw[gray!50!white, line width=0.5pt] (6.4,0.0) to [short] (6.4,8.0);
    \draw[gray!50!white, line width=0.5pt] (8.0,0.0) to [short] (8.0,8.0);


    % generators
    \draw (3.2,-1.8) to [sV, fill=magenta!70!cyan] (3.2,-1.2);
    \draw [short] (3.2,-1.2) to [short] (3.2,-1.0);
    \draw (-1.8,1.6) to [sV, fill=green!50!white] (-1.2,1.6);
    \draw [short] (-1.2,1.6) to [short] (-1.0,1.6);
    \draw (9.2,1.6) to [sV, fill=cyan!50!white] (9.8,1.6);
    \draw [short] (9.2,1.6) to [short] (9.0,1.6);
    \draw (4.8,3.6) to [sV, fill=black!60!white] (4.8,3.0);
    \draw [short] (4.8,3.6) to [short] (4.8,3.8);
    \draw (8.4,4.8) to [sV, fill=yellow!60!white] (9.0,4.8);
    \draw [short] (8.0,4.8) to [short] (8.4,4.8);
    \draw (1.6,9.2) to [sV, fill=orange!60!white] (1.6,9.8);
    \draw [short] (1.6,9.0) to [short] (1.6,9.2);


    % large buses
    \draw[line width=2.5pt] (2.7,0) to [short] (3.7,0);
    \draw[line width=2.5pt] (0.0,1.1) to [short] (0,2.1);
    \draw[line width=2.5pt] (8.0,1.1) to [short] (8,2.1);
    \draw[line width=2.5pt] (1.1,8) to [short] (2.1,8);
    \draw[line width=2.5pt] (2.7,3.2) to [short] (3.7,3.2);
    \draw[line width=2.5pt] (2.7,4.8) to [short] (3.7,4.8);
    \draw[line width=2.5pt] (4.3,6.4) to [short] (5.3,6.4);
    \draw[line width=2.5pt] (0.0,5.9) to [short] (-0.0,6.9);
    \draw[line width=2.5pt] (8,4.3) to [short] (8,5.3);
    \draw[line width=2.5pt] (4.3,4.8) to [short] (5.3,4.8);

    % small buses
    \draw[line width=2.5pt] (2.9,-1) to [short] (3.5,-1);
    \draw[line width=2.5pt] (-1.0,1.3) to [short] (-1,1.9);
    \draw[line width=2.5pt] (9,1.3) to [short] (9,1.9);
    \draw[line width=2.5pt] (4.5,3.8) to [short] (5.1,3.8);
    \draw[line width=2.5pt] (1.3,9.0) to [short] (1.9,9.0);

    \draw[line width=2.5pt] (1.75,3.2) to [short] (1.75,2.6);
    \draw[line width=2.5pt] (1.75,4.8) to [short] (1.75,4.2);
    \draw[line width=2.5pt] (-0.95,6.7) to [short] (-0.95,6.1);
    \draw[line width=2.5pt] (4.5,7.35) to [short] (5.1,7.35);

    % trafos
    \draw (3.2,-1.0) to [voosource] (3.2,0.0);
    \draw (-1,1.6) to [voosource] (0,1.6);
    \draw (8,1.6) to [voosource] (9,1.6);
    \draw (1.6,8) to [voosource] (1.6,9);
    \draw (4.8,4.8) to [voosource] (4.8,3.8);

    \draw (-1,6.4) to [voosource] (0,6.4);
    \draw (4.8,6.4) to [voosource] (4.8,7.4);
    \draw (2.7,4.5) to [voosource] (1.7,4.5);
    \draw (2.7,2.9) to [voosource] (1.7,2.9);

    % loads
    \draw[-{Triangle[length=5mm, width=2mm]}, draw=blue!60!white, fill=blue!60!white] (-1,6.4) -- (-1.6,6.4);
    \draw[-{Triangle[length=5mm, width=2mm]}, draw=blue!60!white, fill=blue!60!white] (4.8,7.4) -- (4.8,8.0);
    \draw[-{Triangle[length=5mm, width=2mm]}, draw=red!60!white, fill=red!60!white] (1.7,4.5) -- (1.1,4.5);
    \draw[-{Triangle[length=5mm, width=2mm]}, draw=blue!60!white, fill=blue!60!white] (1.7,2.9) -- (1.1,2.9);

    \draw (3.0,4.8) to [short] (3.0,4.5);
    \draw (3.0,4.5) to [short] (2.7,4.5);

    \draw (2.75,3.2) to [short] (2.75,2.9);
    \draw (2.75,2.9) to [short] (2.7,2.9);

    % lines
    \draw (3.15,0) to [short] (3.15,3.2);
    \draw (3.25,0) to [short] (3.25,3.2);
    \draw (3.15,3.2) to [short] (3.15,4.8);
    \draw (3.25,3.2) to [short] (3.25,4.8);
    \draw (4.6,4.8) to [short] (4.6,5.1);
    \draw (4.6,5.1) to [short] (3.4,5.1);
    \draw (3.4,5.1) to [short] (3.4,4.8);
    \draw (3.15,4.8) to [short] (3.15,5.1);
    \draw (3.25,4.8) to [short] (3.25,5.1);
    \draw (3.15,5.1) to [short] (4.55,6.1);
    \draw (3.25,5.1) to [short] (4.65,6.1);
    \draw (4.55,6.1) to [short] (4.55,6.4);
    \draw (4.65,6.1) to [short] (4.65,6.4);
    \draw (5.0,6.4) to [short] (5.0,6.1);
    \draw (5.0,6.1) to [short] (7.7,5.1);
    \draw (7.7,5.1) to [short] (8.0,5.1);
    \draw (0,6.4) to [short] (0.3,6.4);
    \draw (0.3,6.4) to [short] (3.0,5.1);
    \draw (3.0,5.1) to [short] (3.0,4.8);


    % storage 2 new lines
    \draw (8,1.6) to [short] (7.7, 1.6);
    \draw (7.7, 1.6) to [short] (3.5,2.9);
    \draw (3.5, 2.9) to [short] (3.5, 3.2);

    \draw (8,1.8) to [short] (7.7, 1.8);
    \draw (7.7, 1.8) to [short] (7.7, 4.4);

    % proposed new lines
    \draw[dashed, draw=red] (8, 4.4) to [short] (3.5,4.4);
    \draw[dashed, draw=red] (3.5,4.4) to [short] (3.5,4.8);
    % \draw[dashed, draw=red] (8,4.4) to [short] (3.5,3.2);
    \draw[dashed, draw=red] (4.5,6.4) to [short] (0,6.4);
    \draw[dashed, draw=red] (3,0) to [short] (3,4.8);
    \draw[dashed, draw=red] (3,3.2) to [short] (0,6.2);
    \draw[dashed, draw=red] (3,0) to [short] (0,6.2);

    % \draw[dashed, draw=red] (8,5.0) to [short] (4.8,5.0);
    % \draw[dashed, draw=red] (4.8,5.0) to [short] (4.8,4.8);
    \draw[dashed, draw=red] (8,4.4) to [short] (3.3,0);

    \draw[draw=red] (6.0,4.25) to [short] (6.3,4.55);
    \draw[draw=red] (6.2,4.25) to [short] (6.5,4.55);

    % \draw[draw=red] (6,4.85) to [short] (6.3,5.15);
    % \draw[draw=red] (6.2,4.85) to [short] (6.5,5.15);

    % \draw[draw=red] (6,4.25) to [short] (6.3,4.55);
    % \draw[draw=red] (6.2,4.25) to [short] (6.5,4.55);

    \draw[draw=red] (2,6.25) to [short] (2.3,6.55);
    \draw[draw=red] (2.2,6.25) to [short] (2.5,6.55);

    \draw[draw=red] (2.85,1.5) to [short] (3.15,1.8);
    \draw[draw=red] (2.85,1.7) to [short] (3.15,2.0);

    \draw[draw=red] (2.3,3.6) to [short] (2.6,3.9);
    \draw[draw=red] (2.2,3.7) to [short] (2.5,4.0);

    \draw[draw=red] (5.3,1.6) to [short] (5.0,1.9);
    \draw[draw=red] (5.4,1.7) to [short] (5.1,2.0);

    % lines for rene and solid interconnection
    \draw (0,6.7) to [short] (0.3,6.7);
    \draw (0,6.8) to [short] (0.3,6.8);
    \draw (0.3,6.7) to [short] (1.65,7.7);
    \draw (0.3,6.8) to [short] (1.55,7.7);
    \draw (1.65,7.7) to [short] (1.65,8.0);
    \draw (1.55,7.7) to [short] (1.55,8.0);

    \draw (0,1.55) to [short] (2.85,1.55);
    \draw (0,1.65) to [short] (2.75,1.65);
    \draw (2.85,1.55) to [short] (2.85,3.2);
    \draw (2.75,1.65) to [short] (2.75,3.2);

    \draw (8,4.9) to [short] (4.9, 4.9);
    \draw (8,5.0) to [short] (4.8, 5.0);
    \draw (4.9,4.9) to [short] (4.9,4.8);
    \draw (4.8,5.0) to [short] (4.8,4.8);

    \draw (8.0,4.4) to [short] (7.7,4.4);
    \draw (7.7,4.4) to [short] (3.5,3.5);
    \draw (3.5,3.5) to [short] (3.5,3.2);

    % legend
    \draw[-{Triangle[length=5mm, width=2mm]}, draw=blue!60!white, fill=blue!60!white] (3.0,8.4) -- (4.0,8.4);
    \draw[-{Triangle[length=5mm, width=2mm]}, draw=red!60!white, fill=red!60!white] (3.0,9.1) -- (4.0,9.1);
    \draw (3.2,9.8) to [sV, fill=magenta!70!cyan] (3.8,9.8);

    \draw (6.0,9.8) to [sV, fill=black!60!white] (6.6,9.8);
    \draw (6.0,9.1) to [sV, fill=yellow!60!white] (6.6,9.1);
    \draw (6.0,8.4) to [sV, fill=green!50!white] (6.6,8.4);

    \draw (9.0,9.8) to [sV, fill=orange!60!white] (9.6,9.8);
    \draw (9.0,9.1) to [sV, fill=cyan!50!white] (9.6,9.1);
    \draw[dashed, draw=red] (9.0,8.4) to [short] (9.5,8.4);

    \node at (4.73, 9.8) {\footnotesize Nuclear};
    \node at (4.66, 9.1) {\footnotesize Load I};
    \node at (4.72, 8.4) {\footnotesize Load II};

    \node at (7.64, 9.8) {\footnotesize Dismantled};
    \node at (7.49, 9.1) {\footnotesize Intercon.};
    \node at (7.28, 8.4) {\footnotesize Wind};

    \node at (10.25, 9.8) {\footnotesize Solar};
    \node at (10.40, 9.1) {\footnotesize Storage};
    \node at (10.45, 8.4) {\footnotesize New line};

    \draw[gray!50!white, line width=0.5pt] (8.5,8.0) to [short] (10.1,8.0);
    \draw[gray!50!white, line width=0.5pt] (8.5,6.4) to [short] (10.1,6.4);
    \draw[gray!50!white, line width=0.5pt] (8.5,6.4) to [short] (8.5,8.0);
    \draw[gray!50!white, line width=0.5pt] (10.1,6.4) to [short] (10.1,8.0);

    \node at (9.30, 6.2) {\footnotesize 50 km};
    \node[rotate=90] at (10.3, 7.2) {\footnotesize 50 km};

    \draw [fill=gray, opacity=0.2, line width=0.01pt] (2.95,10.15) rectangle (11.5,8.05);
    \draw [fill=gray, opacity=0.2, line width=0.01pt] (11.5,8.05) rectangle (8.05,6.05);

    % buses nodes and labels
    \node at (3.6, 0.2) {8};
    \node at (3.6, -1.0) {1};
    \node at (3.6, 4.6) {9};
    \node at (1.75, 5.0) {2};
    \node at (3.9, 3.2) {10};
    \node at (1.75, 3.45) {3};
    \node at (4.4, 6.6) {11};
    \node at (4.4, 7.6) {4};
    \node at (0.0, 7.1) {12};
    \node at (-1.2, 6.7) {5};
    \node at (5.2, 4.6) {13};
    \node at (5.3, 3.7) {7};
    \node at (7.8, 4.6) {6}; 

    \node at (0.9, 8.0) {15};
    \node at (1.1, 9.0) {14};

    \node at (0.0, 0.9) {16};
    \node at (-1.0, 1.1) {17};

    % \node at (8.0, 1.8) {18};
    \node at (8.0, 2.3) {19};
    \node at (9, 2.1) {20};

  \end{circuitikz}

  \caption{Overview of the network with renewables and the potential addition of lines}
  \label{fig:netstorage}
\end{figure}





\section{Storage}
% talk about the storage unit, the P profile we have decided, the technology like Liion, a bit about costs..
Storage devices have been conceived as one of the cornerstone class of elements meant to provide flexibility to smart grids \cite{roberts2011role}. Perhaps in relation to this, electric vehicles (EVs) have been extensively promoted. Apart from the direct impact on decarbonizing mobility, they can act as a source/sink of energy, which also has an appealing influence on power systems \cite{monteiro2011impact}. One of the largest barriers towards adopting batteries has to do with its high investment cost. Although the cost of batteries has been steadily declining for the last two decades \cite{ziegler2021re}, there is still a long way to go. Part of the intention regarding this section is to analyze if storage exerts a positive effect on the power flow of the power system under study. 

Energy storage options can take many forms. For instance, fast-response devices such as supercapacitors provide peaks of power, yet their accumulated energy is rather low. The same characteristics apply to flywheels, although they store the energy mechanically, not electrically. Another option, likely the most favorable one, are batteries. They store the energy chemically, can provide energy for a sustained period of time, and tend to be easily controlled with power converters. This project conceives the storage unit as a lithium-ion battery, since it offers an attractive trade-off between lifetime, power and energy denstiy, and flexibility of operation, despite its high cost.

Table \ref{tab:battery} shows the most notorious characteristics of the chosen lithium-ion battery, along with the corresponding converter. 

\begin{table}[!htb]\centering
  \begin{tabular}{lrr}
    \hline
    \textbf{Magnitude} & \textbf{Units} & \textbf{Value} \\
    \hline
    Capacity & MWh & 50\\
    Peak power & MW & 10\\
    Round-trip battery $\eta$ & \% & 92\\
    Maximum DOD & \% & 80\\
    Converter $\eta$ & \% & 96\\
    Lifetime & cycles & 3000 \\
    Working temperature & \textdegree{C} & -20/55 \\
    Specific energy & Wh/kg & 133 \\
    Orientative LCOE & \texteuro/kWh & 0.11/0.66 \\
    \hline
  \end{tabular}
  \caption{Storage system characteristics. DOD: depth of discharge, LCOE: levelized cost of energy. Data from \cite{diaz2016energy, lo2021efficiency, rahman}.}
  \label{tab:battery}
\end{table}
Having a lifetime of 3000 cycles means that it is not appropriate to charge and discharge the battery several times per day. If that were to be the case, the energy storage system would last for just a few years at most. On the contrary, if the battery is charged and discharged once per day, its lifetime could approach a decade. For this reason, the proposed power profile to follow is represented in Figure \ref{fig:batt}.

\begin{figure}[!htb]\centering
\begin{tikzpicture}
    \begin{axis}[xlabel={Hour}, ylabel={$P_{sto}$ (MW)}, grid=both, grid style={line width=.1pt, draw=gray!10}, major grid style={line width=.2pt,draw=gray!50}, xtick distance = 2, ytick distance = 2, width=12cm, height=5cm, every plot/.append style={very thick}, xmin = 1, xmax = 24, ymin=-9, ymax=6, very thick, grid=both, grid style={line width=.4pt, draw=gray!10}, major grid style={line width=.8pt,draw=gray!50}, legend style={at={(0.8,0.1)},anchor=south west}]
        \addplot[color=black] table[col sep=comma, x=x, y=y] {Data/phase4/battery.csv};
        \addplot[color=black, dashed] table[col sep=comma, x=x, y=y] {Data/phase4/horiz.csv};
\end{axis}

\end{tikzpicture}
\caption{Daily charge and discharge profile for the battery system}
    \label{fig:batt}
  \end{figure}
  This profile resembles the power demand of the loads. This dependence has been established intentionally. In previous analysis, it has been found that during peak hours (around 8 p.m.) the loading of lines is most extreme, and also, the voltages are closer to the lower limit of 0.9~p.u.. Therefore, in general terms, the battery should be discharged during the day and charged at night. According to the chosen sign criteria, positive powers indicate generation, while negative powers denote consumption.

Since the daily demand profile amongst consecutive days do not experience significant differences, it is convenient to always start and finish the day with the same state of charge ($SOC$). To achieve this, the accumulated area under the curve (in Figure \ref{fig:batt}, between the solid line and the dashed line) is kept at $E \cdot DOD=40$~MWh. Besides, the sum of the positive and the negative area yields zero, as desired. 


\section{Dismantled plant}
% mainly discuss carbon emissions of gas, coal... some equations and numbers to estimate, some references, some itemize
Another possibility to consider in order to reduce the dependence on the interconnection consists of employing the once dismantled power plant placed in bus 7 as indicated in Figure \ref{fig:netstorage}. The hypothesis is that this dismantled plant (now rehabilitated) is powered by some kind of fossil fuel (coal, diesel, gas, etc.) or renewable source such as biomass. Thus, even if it could have a positive impact on the electrical magnitudes, its associated environmental impact has to be kept in check. 

We start by exploring the diverse energy sources that could power it to extract conclusions regarding costs, emissions, and geopolitical aspects:
\begin{itemize}
  \item Coal: it is the most polluting of the aforementioned fuels. Even though its emission factor largely depends on the type of coal (lignite, bituminous...), it is in the range of 300 to 360~g CO$_2$/kWh \cite{gencat}. Coal power plants are mostly used in countries with availability of natural resources, which is not the case of Spain.
  \item Diesel: or more generally petroleum, these power plants are more a rarity than an actual common choice. They are suitable to be installed in microgrids or islanded systems thank to their flexibility and suitability for hybrid systems where they act as backup units \cite{ismail2013techno}. Since their chemical proportion between hydrogen and carbon is larger than in the case of coal, its emission factor becomes lower; it is around 270~g CO$_2$/kWh \cite{gencat}. 
  \item Natural gas: it is cleaner than the two previous fuels, with an emission factor of 180~g CO$_2$/kWh approximately \cite{gencat}. Power plants powered by natural gas usually take the form of the so-called combined cycle gas turbine (CCGT) power plant. During the previous decades, Spain promoted the installation of combined cycle power plants, which nowadays are unused for the most part \cite{aleasoft}. Probably the largest disadvantage of these power plants is the high cost of natural gas, which has fluctuated extensively during the last year \cite{gascost}. 
  \item Biomass: it is considered a renewable source with carbon neutral emissions \cite{gencat}, since the CO$_2$ emitted during combustion has previously been absorbed from the atmosphere. Then, the emission factor of biomass is 0~gCO$_2$/kWh. Nowadays, there are already some examples where the existing facilities have been converted from coal or natural gas to biomass \cite{grontmij}.
\end{itemize}
When it comes to the implementation of the program, it makes no difference to consider one technology or another, as only the output power is relevant. However, this chapter also focuses on the emissions. Therefore, the results section analyzes the results for coal, diesel and natural gas.

The chosen power generation profile for the rehabilitated plant is depicted in Figure \ref{fig:dism}. Compared to storage or the demand, here the peaks are more exaggerated. The underlying idea is that the system is more in need of generation during peak hours. And if demand increases, generation should increase by a larger factor to keep the losses and loadings of the lines at satisfactory values. A peak power of 50~MW has been considered, which corresponds to the most critical hour of the day. The plant is supposed to stop operating at night, and its voltage is maintained at 1.02~p.u. since it acts as a PV node.



\begin{figure}[!htb]\centering
\begin{tikzpicture}
    \begin{axis}[xlabel={Hour}, ylabel={$P_{dism}$ (MW)}, grid=both, grid style={line width=.1pt, draw=gray!10}, major grid style={line width=.2pt,draw=gray!50}, xtick distance = 2, ytick distance = 10, width=12cm, height=5cm, every plot/.append style={very thick}, xmin = 1, xmax = 24, ymin=-1, ymax=55, very thick, grid=both, grid style={line width=.4pt, draw=gray!10}, major grid style={line width=.8pt,draw=gray!50}, legend style={at={(0.8,0.1)},anchor=south west}]
        \addplot[color=black] table[col sep=comma, x=x, y=y] {Data/phase4/dism.csv};
\end{axis}
\end{tikzpicture}
\caption{Daily generation profile of the rehabilitated plant}
    \label{fig:dism}
  \end{figure}

With the storage and the generation of the rehabilitated plant defined, the simulation can be ran again. The next sections explore the results and draw conclusions from them.


\section{Base case results}
% such as in the net picture, compare with previous results (table with 3 columns, 1 for each case). Copy the previous table, just add 1 column
% also emissions for the 3 types of plants. 332 MWh daily from the dismantled, calculate the daily emissions also considering the slack
The base case considers the grid as shown in Figure \ref{fig:netstorage}, with the black lines representing actual connections. The red lines are only taken into account for the contingency analysis. In any case, it is expected that the presence of the battery and an additonal plant should contribute to improving the electrical magnitudes, at least slightly. Table \ref{tab:compare} shows this is precisely the case.

\begin{table}[!htb]\centering
  \begin{tabular}{lrrr}
    \hline
    \textbf{Attributes} & \textbf{Phase 2} & \textbf{Phase 3} & \textbf{Phase 4}\\
    \hline
    $V_{min}$ (p.u.) & 0.962 & 0.968 & 0.971 \\
    $V_{max}$ (p.u.) & 1.050 & 1.050 & 1.050 \\
    Max. load (\%) & 41.65 & 40.76 & 39.63 \\
    Max. losses (MW) & 14.54 & 14.43 & 13.89 \\
    Correct operation? & Yes & Yes & Yes \\
    \hline
  \end{tabular}
  \caption{Main results to compare between the grid without renewables, with renewables, and with storage and the rehabilitated plant}
  \label{tab:compare2}
\end{table}
The results tend to be the ones we could have expected. The minimum voltage rises a bit thanks to the inclusion of more generation sources, which control the voltages at their respective buses and hence ensure the operation around nominal voltages. Similarly, the loadings of the lines are reduced a bit, being the most loaded line at only 39.63\%. There could be reasons to think that the system is oversized, because lines do not approach their limits of 80\%. While these claims would be full of validity, faults have to be taken into account as well. Redundant connections are a must if the criteria $N-1$ has to be met. 

The maximum losses are again reduced. When the results from the three phases are compared, a clear pattern emerges. Going from phase 2 to phase 4, results improve each time the grid becomes stronger and more decentralized. These are in essence some of the core characteristics of smart grids. Another central goal of smart grids has to do with decarbonizing the system, which is described below. 
% expected emissions here, not in contingencies

Regarding emissions, it is known beforehand that storage, renewables and nuclear do not emit CO$_2$ gases or equivalent during their operation. The only generation sources to account for emissions are the interconnection and the dismantled plant. In this base case, the dismantled plant produced 332~MWh, whereas the interconnection provided 4686~MWh. The interconnection can be assumed to have an emission factor of 190~kg CO$_2$-eq/MWh, which is the average for Spain \cite{ree_co2, spork2015increasing}. With this, the total emissions for a full-day operation are gathered in Table \ref{tab:emit} depending on the technology of the dismantled power plant. 

\begin{table}[!htb]\centering
  \begin{tabular}{lrrr}
    \hline
    \textbf{Fuel} & \textbf{Dismantled} & \textbf{Interconnection} & \textbf{Total} \\
    & (tCO$_2$-eq) &  (tCO$_2$-eq)  &  (tCO$_2$-eq) \\
    \hline
    Coal & 108.56 & 890.34 & 998.90 \\
    Diesel & 89.64 & 890.34 & 979.98 \\
    Gas & 59.76 & 890.34 & 950.10 \\
    Biomass & 0.00 & 890.34 & 890.34 \\
    \hline
  \end{tabular}
  \caption{Total daily emissions depending on the scenario}
  \label{tab:emit}
\end{table}
The conclusion we reach is that even if there is a non-negligible difference in the emissions caused by the dismantled plant, its overall influence is not that relevant. The interconnection accounts for 90 to 95\% of the total CO$_2$ emissions. The only counteract to that is the integration of nuclear and renewable power plants, which produce about 10.06~GWh daily. This is about twice the energy imported from the slack bus, so in proportion, about 2/3 of the total energy come from clean sources. This is equivalent to saying that the electricity emissions factor of the whole system is roughly 63~kg CO$_2$-eq/MWh. This is a more than acceptable value that most European countries do not reach (see \cite{co2iea}).

Finally, an economic comparison can be performed according to the related costs of each energy source. These costs include the net present value of the unit-cost of electricity over the lifetime of a system, known as Levelized Cost of Electricity (LCOE), and the CO$_2$ emission costs. Considering the LCOE of each energy source \cite{irena, roussanaly2020techno, freeing, lazard} and the current CO$_2$ price \cite{ember}, the biomass power plant is the cheapest option since it has no CO$_2$ emission costs associated. It is 16.32\% cheaper than natural gas, followed by coal with 35.97\%, and diesel with 65.21\%.

\section{Contingency analysis}
% similar as before, top 10
Just as in previous phases, a contingency analysis is performed to verify if new lines are required, and if so, which ones should be. The marked red lines shown in Figure \ref{fig:netstorage} are the potentially added lines, while the black ones are the already established ones. Table \ref{tab:top10_stor} presents the top 10 best configurations under faults. 

\begin{table}[!htb]\centering
  \begin{tabular}{ccc}
    \hline
    \textbf{Identifier} & \textbf{New lines} & \textbf{Infraestructure cost (M\texteuro)}\\
    \hline
    % 24 & [8-9, 10-12] & 402.71 \\
    % 12 & [8-9, 9-6] & 406.21 \\
    % 46 & [11-12, 8-9] & 406.21 \\
    % 8 & [8-9, 8-12] & 409.54 \\
    % 0 & [11-12, 8-9, 10-12] & 463.84 \\
    % 4 & [8-9, 10-12, 9-6] & 463.84 \\
    % 37 & [11-12, 10-12, 9-6] & 463.84 \\
    % 63 & [8-9, 10-12, 8-12] & 467.18 \\
    % 20 & [11-12, 8-9, 9-6] & 467.34 \\
    % 38 & [11-12, 8-9, 8-12] & 470.67 \\
    33 & [8-9] & 419.49 \\
    24 & [8-9, 10-12] & 477.12 \\
    12 & [8-9, 9-6] & 480.62 \\
    46 & [11-12, 8-9] & 480.62 \\
    8 & [8-9, 8-12] & 483.96 \\
    57 & [8-9, 6-8] & 505.94 \\
    0 & [8-9, 10-12] & 538.25 \\
    4 & [8-9, 10-12, 9-6] & 538.25 \\
    37 & [11-12, 10-12, 8-12] & 538.25 \\
    63 & [8-9, 10-12, 8-12] & 541.59 \\
    \hline
  \end{tabular}
  \caption{Best configurations with the additional lines with storage and a rehabilitated plant}
  \label{tab:top10_stor}
\end{table}
The results indicate that, contrary to previous configurations, installing only the additional line from buses 8 to 9 is enough to meet the $N-1$ criteria. This is certainly a positive result, because in all other cases, at least two lines had to be added. Notice, however, that most configurations are the same as in Table \ref{tab:top10_rene}. This makes sense since the topology is basically identical, with just minor variations due to the added storage and the rehabilitated plant.

The total installation cost increases a bit respect to the case where only renewables were added. Of course, new lines have been included to connect the storage unit to the system. Yet, the conclusion has not changed. Moving towards a smart grid, which includes a larger penetration of renewables, batteries, and other elements, supposes a large investment cost. The question is if this invesment ends up being worth it. Multiple reasons could be added into the equation, but what is for sure is that there is a climate emergency to combat. 


% only installing line 8-9 is enough! so only 1 more line needed!! good news





