\begin{lstlisting}[caption={Main code in Python with the Pandapower library}]
import pandapower as pp
import pandas as pd
import numpy as np
import pandapower.control as control
import pandapower.networks as nw
import pandapower.timeseries as timeseries
from pandapower.timeseries.data_sources.frame_data import DFData
from pandapower.plotting import simple_plot

from line_param_calc import calc_line



pd.set_option('display.max_rows', 500)
pd.set_option('display.max_columns', 500)
pd.set_option('display.width', 1000)

def initialize_net(path_bus, path_geodata, path_line, path_demand, path_busload, path_generation, path_busgen, path_trafo):
    """
    initialize the grid from the .csv files

    :param path_bus: path to the bus .csv file
    :param geodata: path to the geodata .csv file
    :param path_line: path to the line .csv file
    :param path_demand: path to the normalized demand .csv file
    :param busload: path to the bus-load look up table .csv file
    :param path_generation: path to the normalized generation .csv file
    :param busgen: path to the bus-generator look up table .csv file
    :param trafo: path to the trafo .csv file
    :return: the net class
    """

    def create_bus(path_bus, path_geodata):
        """
        adapts the data from the bus file (if needed)

        :param path_bus:
        :param path_geodata:
        :return: the net with the buses added
        """

        df_bus = pd.read_csv(path_bus)
        df_geodata = pd.read_csv(path_geodata)

        net.bus = df_bus

        # adapt geodata
        for ll in range(len(df_geodata)):
            indx_bus = pp.get_element_index(net, "bus", df_geodata['name'][ll])
            df_geodata['name'][ll] = indx_bus

        net.bus_geodata = df_geodata

        return net


    def create_line(path_line):
        """
        adapts the data from the line file

        :param path_line:
        :return: the net with the lines added
        """

        df_line = pd.read_csv(path_line)
        for _, line in df_line.iterrows():
            from_bus = pp.get_element_index(net, "bus", line.from_bus)
            to_bus = pp.get_element_index(net, "bus", line.to_bus)

            rr, xx, cc, imax = calc_line(line.a,
                                         line.b,
                                         line.c,
                                         line.d,
                                         line.e,
                                         line.max_i,
                                         int(line.parallel))

            pp.create_line_from_parameters(net,
                                           from_bus,
                                           to_bus,
                                           length_km=line.length,
                                           r_ohm_per_km=rr,
                                           x_ohm_per_km=xx,
                                           c_nf_per_km=cc,
                                           max_i_ka=imax,
                                           name=line.name_l,
                                           parallel=line.parallel)

        return net


    def create_load(path_demand, path_busload, path_bus):
        """
        adapts the load files

        :param path_demand:
        :param path_busload:
        :param path_bus:
        :return: the net with the loads added
        """

        df_demand = pd.read_csv(path_demand)
        df_busload = pd.read_csv(path_busload)
        df_bus = pd.read_csv(path_bus)

        # create basic load dataframe
        # find the bus index of each load
        load_indx = []
        for _, load in df_busload.iterrows():
            bus_load = pp.get_element_index(net, "bus", load.bus)
            load_indx.append(bus_load)

        load_indx = pd.DataFrame(load_indx)
        load_indx = load_indx.rename(columns={0: "bus"})

        # load name and peak power
        load_name = df_busload['bus']
        load_pmw = df_busload['p_mw']
        load_qmvar = df_busload['q_mvar']

        # merge in a full dataframe
        headers = ["name", "bus", "p_mw", "q_mvar"]
        df_load = pd.concat([load_name, load_indx, load_pmw, load_qmvar], axis=1)
        df_load.columns.values[0] = "name"

        # create time series from the basic load df
        Nt = len(df_demand)
        Nl = len(df_load)
        pmw_ts = np.zeros((Nt, Nl), dtype=float)
        qmvar_ts = np.zeros((Nt, Nl), dtype=float)
        for i in range(Nt):  # number of time periods
            pmw_ts[i,:] = df_load['p_mw'][:] * df_demand['norm'][i]
            qmvar_ts[i,:] = df_load['q_mvar'][:] * df_demand['norm'][i]

        # form loads as a static picture (initial time)
        for ll in range(len(df_busload)):
            pp.create_load(net, bus=load_indx['bus'][ll], p_mw=pmw_ts[0, ll], q_mvar=qmvar_ts[0, ll], name=load_name[ll], index=int(ll))

        # timeseries
        df_pload_ts = pd.DataFrame(pmw_ts, index=list(range(Nt)), columns=net.load.index)
        df_qload_ts = pd.DataFrame(qmvar_ts, index=list(range(Nt)), columns=net.load.index)
        ds_pload_ts = DFData(df_pload_ts)
        ds_qload_ts = DFData(df_qload_ts)
        const_load = control.ConstControl(net, element='load', element_index=net.load.index, variable='p_mw', data_source=ds_pload_ts, profile_name=net.load.index)
        const_load = control.ConstControl(net, element='load', element_index=net.load.index, variable='q_mvar', data_source=ds_qload_ts, profile_name=net.load.index)  # add the reactive like this?

        return net


    def create_generator(path_generation, path_busgen, path_bus):
        """
        adapts the generation files

        :param path_generation:
        :param path_busgenerator:
        :param path_bus:
        :return: the net with the generators added
        """

        df_generation = pd.read_csv(path_generation)
        df_busgen = pd.read_csv(path_busgen)
        df_bus = pd.read_csv(path_bus)

        # create basic generator dataframe
        # find the bus index of each gen
        gen_indx = []
        for _, gen in df_busgen.iterrows():
            bus_gen = pp.get_element_index(net, "bus", gen.bus)
            gen_indx.append(bus_gen)

        gen_indx = pd.DataFrame(gen_indx)
        gen_indx = gen_indx.rename(columns={0: "bus"})

        # load name and peak power
        gen_name = df_busgen['bus']
        gen_pmw = df_busgen['p_mw']
        gen_vpu = df_busgen['vm_pu']

        # merge in a full dataframe
        headers = ["name", "bus", "p_mw", "vm_pu"]
        df_gen = pd.concat([gen_name, gen_indx, gen_pmw, gen_vpu], axis=1)
        df_gen.columns.values[0] = "name"

        # create time series from the basic load df
        Nt = len(df_generation)
        Ng = len(df_gen)
        pmw_ts = np.zeros((Nt, Ng), dtype=float)
        for i in range(Nt):  # number of time periods
            pmw_ts[i,:] = df_gen['p_mw'][:] * df_generation['norm'][i]

        # gen structure for 1 t
        for ll in range(len(df_busgen)):
            pp.create_gen(net, bus=gen_indx['bus'][ll], p_mw=pmw_ts[0, ll], vm_pu=gen_vpu[ll], name=gen_name[ll], index=int(ll))


        # timeseries
        df_gen_ts = pd.DataFrame(pmw_ts, index=list(range(Nt)), columns=net.gen.index)
        ds_gen_ts = DFData(df_gen_ts)
        const_gen = control.ConstControl(net, element='gen', element_index=net.gen.index, variable='p_mw', data_source=ds_gen_ts, profile_name=net.gen.index)

        return net


    def create_intercon(path_bus):
        """
        defines the interconnection (slack bus)

        :param path_bus:
        :return: the net with the interconnection added
        """

        df_bus = pd.read_csv(path_bus)

        # find the slack index
        slack_indx = 0
        for ll in range(len(df_bus)):
            # slack_indx = pp.get_element_index(net, "bus", bb.name)
            if df_bus['name'][ll] == 'intercon':
                slack_indx = pp.get_element_index(net, "bus", df_bus['name'][ll])

        pp.create_ext_grid(net, slack_indx, vm_pu=1.0, va_degree=0)

        return net


    def create_trafo(path_trafo):
        """
        defines the transformers

        :param path_trafo:
        :return: the net with the transformers added
        """

        df_trafo = pd.read_csv(path_trafo)

        # for trafo in df_trafo:
        for _, trafo in df_trafo.iterrows():
            hv_bus = pp.get_element_index(net, "bus", trafo.hv_bus)
            lv_bus = pp.get_element_index(net, "bus", trafo.lv_bus)

            pp.create_transformer_from_parameters(net,
                                                  hv_bus,
                                                  lv_bus,
                                                  trafo.sn_mva,
                                                  trafo.vn_hv_kv,
                                                  trafo.vn_lv_kv,
                                                  trafo.vkr_percent,
                                                  trafo.vk_percent,
                                                  trafo.pfe_kw,
                                                  trafo.i0_percent)

        return net






    # create empty network
    net = pp.create_empty_network()

    # buses
    net = create_bus(path_bus, path_geodata)

    # lines
    net = create_line(path_line)

    # loads
    net = create_load(path_demand, path_busload, path_bus)

    # gens
    net = create_generator(path_generation, path_busgen, path_bus)

    # interconnection
    net = create_intercon(path_bus)

    # trafos
    net = create_trafo(path_trafo)



    return net


if __name__ == "__main__":
    # load paths
    path_bus = 'Datafiles/bus1.csv'
    path_geodata = 'Datafiles/geodata1.csv'
    path_line = 'Datafiles/line1.csv'
    path_demand = 'Datafiles/demand1.csv'
    path_busload = 'Datafiles/bus_load1.csv'
    path_generation = 'Datafiles/generation1.csv'
    path_busgen = 'Datafiles/bus_gen1.csv'
    path_trafo = 'Datafiles/trafo1.csv'

    # define net
    net = initialize_net(path_bus, path_geodata, path_line, path_demand, path_busload, path_generation, path_busgen, path_trafo)

    # run timeseries
    ow = timeseries.OutputWriter(net, output_path="./Results/", output_file_type=".xlsx")
    ow.log_variable('res_bus', 'vm_pu')
    ow.log_variable('res_line', 'loading_percent')
    timeseries.run_timeseries(net)

    # run diagnostic
    # pp.diagnostic(net)
    print(net.bus)

    # plot
    # pp.plotting.simple_plot(net)
    # simple_plot(net)
\end{lstlisting}


\begin{lstlisting}[caption={Code for the calculation of lines}]
import numpy as np

def calc_line(a, b, c, d, e, immax, npar, Rca, Dext, kgg):
    """
    calculate r, x, c, and return also Imax

    :param a: horizontal distance between A1 and C2
    :param b: horizontal distance between B1 and B2
    :param c: horizontal distance between C1 and A2
    :param d: vertical distance between A1 and B1
    :param e: vertical distance between B1 and C1
    :param immax: max current in A
    :param npar: number of parallel lines (1 or 2)
    :param Rca: ac resistance in ohm/km
    :param Dext: external diameter in mm
    :param kg: factor of roughly 0.8
    :return: r, x, c, imax
    """

    def single_line(a, b, immax, Rca, Dext, kgg):
        """
        calculate the R, X, C parameters, also return Imax

        :param a: horizontal distance between A and C
        :param b: vertical distance between A and B
        :param immax: max current in A
        :param Rca: ac resistance in ohm/km
        :param Dext: external diameter in mm
        :param kg: factor of roughly 0.8

        :return: R, X, C, Imax, in the units desired by pandapower
        """

        # cardinal: https://www.elandcables.com/media/38193/acsr-astm-b-aluminium-conductor-steel-reinforced.pdf
        # 54 Al + 7 St, Imax = 888.98 A

        w = 2 * np.pi * 50  # rad / s
        Imax = immax * 1e-3  # kA
        # Stot = 547.3 * 1e-6  # m2, the total section
        # R_ac_75 = 0.07316 * 1e-3  # ohm / m
        # kg = 0.809  # from the slides in a 54 + 7

        R_ac_75 = Rca * 1e-3  # ohm / m, should we correct by temperatures?
        Stot = np.pi * Dext ** 2 / 4 * 1e-6  # m2, the total section
        kg = kgg

        r = np.sqrt(Stot / np.pi)  # considering the total section

        dab = np.sqrt((a / 2) ** 2 + b ** 2)
        dbc = np.sqrt((a / 2) ** 2 + b ** 2)
        dca = a

        GMD = (dab * dbc * dca) ** (1 / 3)
        GMR = kg * r
        RMG = r

        L = 4 * np.pi * 1e-7 / (2 * np.pi) * np.log(GMD / GMR)  # H / m

        C = 2 * np.pi * 1e-9 / (36 * np.pi) / np.log(GMD / RMG)  # F / m

        # in the units pandapower wants
        R_km = R_ac_75 * 1e3  # ohm / km
        X_km = L * w * 1e3  # ohm / km
        C_km = C * 1e9 * 1e3  # nF / km

        return R_km, X_km, C_km, Imax


    def double_line(a, b, c, d, e, immax, Rca, Dext, kgg):
        """
        calculate the R, X, C parameters, also return Imax

        :param a: horizontal distance between A1 and C2
        :param b: horizontal distance between B1 and B2
        :param c: horizontal distance between C1 and A2
        :param d: vertical distance between A1 and B1
        :param e: vertical distance between B1 and C1
        :param immax: max current in A
        :param Rca: ac resistance in ohm/km
        :param Dext: external diameter in mm
        :param kgg: factor of roughly 0.8
        :return: R, X, C, Imax, in the units desired by pandapower
        """

        # cardinal: https://www.elandcables.com/media/38193/acsr-astm-b-aluminium-conductor-steel-reinforced.pdf
        # 54 Al + 7 St, Imax = 888.98 A

        w = 2 * np.pi * 50  # rad / s
        Imax = immax * 1e-3 * 2  # kA, for the full line, x2
        # Stot = 547.3 * 1e-6  # m2, the total section
        # R_ac_75 = 0.07316 * 1e-3  # ohm / m
        # kg = 0.809  # from the slides in a 54 + 7

        R_ac_75 = Rca * 1e-3  # ohm / m, should we correct by temperatures?
        Stot = np.pi * Dext ** 2 / 4 * 1e-6  # m2, the total section
        kg = kgg



        r = np.sqrt(Stot / np.pi)  # considering the total section

        da1b1 = np.sqrt((b / 2 - a / 2) ** 2 + d ** 2)
        da1b2 = np.sqrt((a / 2 + b / 2) ** 2 + d ** 2)
        da2b1 = np.sqrt((c / 2 + b / 2) ** 2 + e ** 2)
        da2b2 = np.sqrt((b / 2 - c / 2) ** 2 + e ** 2)

        db1c1 = np.sqrt((b / 2 - c / 2) ** 2 + e ** 2)
        db1c2 = np.sqrt((b / 2 + a / 2) ** 2 + d ** 2)
        db2c1 = np.sqrt((b / 2 + c / 2) ** 2 + e ** 2)
        db2c2 = np.sqrt((b / 2 - a / 2) ** 2 + d ** 2)

        dc1a1 = np.sqrt((a / 2 - c / 2) ** 2 + (d + e) ** 2)
        dc1a2 = c
        dc2a1 = a
        dc2a2 = np.sqrt((a / 2 - c / 2) ** 2 + (d + e) ** 2)

        dab = (da1b1 * da1b2 * da2b1 * da2b2) ** (1 / 4)
        dbc = (db1c1 * db1c2 * db2c1 * db2c2) ** (1 / 4)
        dca = (dc1a1 * dc1a2 * dc2a1 * dc2a2) ** (1 / 4)

        rp = kg * r

        da1a2 = np.sqrt((a / 2 + c / 2) ** 2 + (d + e) ** 2)
        db1b2 = b
        dc1c2 = np.sqrt((c / 2 + a / 2) ** 2 + (d + e) ** 2)

        drap = np.sqrt(rp * da1a2)
        drbp = np.sqrt(rp * db1b2)
        drcp = np.sqrt(rp * dc1c2)

        dra = np.sqrt(r * da1a2)
        drb = np.sqrt(r * db1b2)
        drc = np.sqrt(r * dc1c2)

        GMD = (dab * dbc * dca) ** (1 / 3)
        GMR = (drap * drbp * drcp) ** (1 / 3)
        RMG = (dra * drb * drc) ** (1 / 3)

        L = 4 * np.pi * 1e-7 / (2 * np.pi) * np.log(GMD / GMR)  # H / m

        C = 2 * np.pi * 1e-9 / (36 * np.pi) / np.log(GMD / RMG)  # F / m

        # in the units pandapower wants
        R_km = R_ac_75 / 2 * 1e3  # ohm / km, like 2 resistances in parallel
        X_km = L * w * 1e3  # ohm / km
        C_km = C * 1e9 * 1e3  # nF / km

        return R_km, X_km, C_km, Imax

    if npar == 1:
        rr, xx, cc, imm = single_line(a, b, immax, Rca, Dext, kgg)
    elif npar == 2:
        rr, xx, cc, imm = double_line(a, b, c, d, e, immax, Rca, Dext, kgg)
    else:
        print('Error: number of parallel lines is not 1 nor 2')

    return rr, xx, cc, imm

# rr, xx, cc, ii = double_line(11, 2, 4, 5, 6, 1000)
# print(rr, xx, cc, ii)

\end{lstlisting}
