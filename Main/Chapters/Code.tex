\begin{lstlisting}[caption={Main code in Python with the Pandapower library}]
import pandapower as pp
import pandas as pd
import numpy as np
import pandapower.control as control
import pandapower.networks as nw
import pandapower.timeseries as timeseries
import itertools
from pandapower.timeseries.data_sources.frame_data import DFData
from pandapower.plotting import simple_plot
import time

import sys
import codecs

from line_param_calc import calc_line


pd.set_option('display.max_rows', 500)
pd.set_option('display.max_columns', 500)
pd.set_option('display.width', 1000)

def initialize_net(path_bus, path_geodata, path_line, path_demand, path_busload, path_generation, path_busgen, path_trafo, rene=False, path_solar_prof=None, path_wind_prof=None):
    """
    initialize the grid from the .csv files

    :param path_bus: path to the bus .csv file
    :param geodata: path to the geodata .csv file
    :param path_line: path to the line .csv file
    :param path_demand: path to the normalized demand .csv file
    :param busload: path to the bus-load look up table .csv file
    :param path_generation: path to the normalized generation .csv file
    :param busgen: path to the bus-generator look up table .csv file
    :param trafo: path to the trafo .csv file
    :param rene: boolean for renewables
    :return: the net class
    """

    def create_bus(path_bus, path_geodata):
        """
        adapts the data from the bus file (if needed)

        :param path_bus:
        :param path_geodata:
        :return: the net with the buses added
        """

        df_bus = pd.read_csv(path_bus)
        df_geodata = pd.read_csv(path_geodata)

        net.bus = df_bus

        # adapt geodata
        for ll in range(len(df_geodata)):
            indx_bus = pp.get_element_index(net, "bus", df_geodata['name'][ll])
            df_geodata['name'][ll] = indx_bus

        net.bus_geodata = df_geodata

        return net


    def create_line(path_line):
        """
        adapts the data from the line file

        :param path_line:
        :return: the net with the lines added
        """

        df_line = pd.read_csv(path_line)
        for _, line in df_line.iterrows():
            from_bus = pp.get_element_index(net, "bus", line.from_bus)
            to_bus = pp.get_element_index(net, "bus", line.to_bus)

            rr, xx, cc, imax = calc_line(line.a,
                                         line.b,
                                         line.c,
                                         line.d,
                                         line.e,
                                         line.max_i,
                                         int(line.parallel),
                                         line.Rca,
                                         line.Dext,
                                         line.kg)

            pp.create_line_from_parameters(net,
                                           from_bus,
                                           to_bus,
                                           length_km=line.length,
                                           r_ohm_per_km=rr,
                                           x_ohm_per_km=xx,
                                           c_nf_per_km=cc,
                                           max_i_ka=imax,
                                           name=line.name_l,
                                           parallel=line.parallel,
                                           in_service=line.in_service)

        return net


    def create_load(path_demand, path_busload, path_bus):
        """
        adapts the load files

        :param path_demand:
        :param path_busload:
        :param path_bus:
        :return: the net with the loads added
        """

        df_demand = pd.read_csv(path_demand)
        df_busload = pd.read_csv(path_busload)
        df_bus = pd.read_csv(path_bus)

        # create basic load dataframe
        # find the bus index of each load
        load_indx = []
        for _, load in df_busload.iterrows():
            bus_load = pp.get_element_index(net, "bus", load.bus)
            load_indx.append(bus_load)

        load_indx = pd.DataFrame(load_indx)
        load_indx = load_indx.rename(columns={0: "bus"})

        # load name and peak power
        load_name = df_busload['bus']
        load_pmw = df_busload['p_mw']
        load_qmvar = df_busload['q_mvar']

        # merge in a full dataframe
        headers = ["name", "bus", "p_mw", "q_mvar"]
        df_load = pd.concat([load_name, load_indx, load_pmw, load_qmvar], axis=1)
        df_load.columns.values[0] = "name"

        # create time series from the basic load df
        Nt = len(df_demand)
        Nl = len(df_load)
        pmw_ts = np.zeros((Nt, Nl), dtype=float)
        qmvar_ts = np.zeros((Nt, Nl), dtype=float)
        for i in range(Nt):  # number of time periods
            pmw_ts[i,:] = df_load['p_mw'][:] * df_demand['norm'][i]
            qmvar_ts[i,:] = df_load['q_mvar'][:] * df_demand['norm'][i]

        # form loads as a static picture (initial time)
        for ll in range(len(df_busload)):
            pp.create_load(net, bus=load_indx['bus'][ll], p_mw=pmw_ts[0, ll], q_mvar=qmvar_ts[0, ll], name=load_name[ll], index=int(ll))

        # timeseries
        df_pload_ts = pd.DataFrame(pmw_ts, index=list(range(Nt)), columns=net.load.index)
        df_qload_ts = pd.DataFrame(qmvar_ts, index=list(range(Nt)), columns=net.load.index)
        ds_pload_ts = DFData(df_pload_ts)
        ds_qload_ts = DFData(df_qload_ts)
        const_load = control.ConstControl(net, element='load', element_index=net.load.index, variable='p_mw', data_source=ds_pload_ts, profile_name=net.load.index)
        const_load = control.ConstControl(net, element='load', element_index=net.load.index, variable='q_mvar', data_source=ds_qload_ts, profile_name=net.load.index)  # add the reactive like this?

        return net


    def create_generator(path_generation, path_busgen, path_bus):
        """
        adapts the generation files

        :param path_generation:
        :param path_busgenerator:
        :param path_bus:
        :return: the net with the generators added
        """

        df_generation = pd.read_csv(path_generation)
        df_busgen = pd.read_csv(path_busgen)
        df_bus = pd.read_csv(path_bus)

        # create basic generator dataframe
        # find the bus index of each gen
        gen_indx = []
        for _, gen in df_busgen.iterrows():
            bus_gen = pp.get_element_index(net, "bus", gen.bus)
            gen_indx.append(bus_gen)

        gen_indx = pd.DataFrame(gen_indx)
        gen_indx = gen_indx.rename(columns={0: "bus"})

        # load name and peak power
        gen_name = df_busgen['bus']
        gen_pmw = df_busgen['p_mw']
        gen_vpu = df_busgen['vm_pu']

        # merge in a full dataframe
        headers = ["name", "bus", "p_mw", "vm_pu"]
        df_gen = pd.concat([gen_name, gen_indx, gen_pmw, gen_vpu], axis=1)
        df_gen.columns.values[0] = "name"

        # create time series from the basic load df
        Nt = len(df_generation)
        Ng = len(df_gen)
        pmw_ts = np.zeros((Nt, Ng), dtype=float)
        for i in range(Nt):  # number of time periods
            pmw_ts[i,:] = df_gen['p_mw'][:] * df_generation['norm'][i]

        # gen structure for 1 t
        for ll in range(len(df_busgen)):
            pp.create_gen(net, bus=gen_indx['bus'][ll], p_mw=pmw_ts[0, ll], vm_pu=gen_vpu[ll], name=gen_name[ll], index=int(ll))


        # timeseries
        df_gen_ts = pd.DataFrame(pmw_ts, index=list(range(Nt)), columns=net.gen.index)
        ds_gen_ts = DFData(df_gen_ts)
        const_gen = control.ConstControl(net, element='gen', element_index=net.gen.index, variable='p_mw', data_source=ds_gen_ts, profile_name=net.gen.index)

        return net


    def create_generator_rene(path_generation, path_busgen, path_bus):
        """
        adapts the generation files

        :param path_generation:
        :param path_busgenerator:
        :param path_bus:
        :return: the net with the generators added
        """

        df_generation = pd.read_csv(path_generation)
        df_busgen = pd.read_csv(path_busgen)
        df_bus = pd.read_csv(path_bus)
        # df_solar_prof = pd.read_csv(path_solar_profile)
        # df_wind_prof = pd.read_csv(path_wind_profile)

        # create basic generator dataframe
        # find the bus index of each gen
        gen_indx = []
        for _, gen in df_busgen.iterrows():
            bus_gen = pp.get_element_index(net, "bus", gen.bus)
            gen_indx.append(bus_gen)

        gen_indx = pd.DataFrame(gen_indx)
        gen_indx = gen_indx.rename(columns={0: "bus"})

        # load name and peak power
        gen_name = df_busgen['bus']
        gen_pmw = df_busgen['p_mw']
        gen_vpu = df_busgen['vm_pu']

        # merge in a full dataframe
        headers = ["name", "bus", "p_mw", "vm_pu"]
        df_gen = pd.concat([gen_name, gen_indx, gen_pmw, gen_vpu], axis=1)
        df_gen.columns.values[0] = "name"


        # create time series from the basic load df
        Nt = len(df_generation)
        Ng = len(df_gen)


        pmw_ts = np.zeros((Nt, Ng), dtype=float)
        for gg in range(Ng):
            for i in range(Nt):  # number of time periods
                # pmw_ts[i,gg] = df_gen['p_mw'][gg] * df_generation['norm'][i]
                pmw_ts[i,gg] = df_gen['p_mw'][gg] * df_generation.iloc[i,gg+1]
            
            pp.create_gen(net, bus=gen_indx['bus'][gg], p_mw=pmw_ts[0, gg], vm_pu=gen_vpu[gg], name=gen_name[gg], index=int(gg))  # take t=0

            # gen structure for 1 t
            # for ll in range(len(df_busgen)):
                # pp.create_gen(net, bus=gen_indx['bus'][ll], p_mw=pmw_ts[0, ll], vm_pu=gen_vpu[ll], name=gen_name[ll], index=int(ll))


        # timeseries
        df_gen_ts = pd.DataFrame(pmw_ts, index=list(range(Nt)), columns=net.gen.index)
        ds_gen_ts = DFData(df_gen_ts)
        const_gen = control.ConstControl(net, element='gen', element_index=net.gen.index, variable='p_mw', data_source=ds_gen_ts, profile_name=net.gen.index)

        return net


    def create_intercon(path_bus):
        """
        defines the interconnection (slack bus)

        :param path_bus:
        :return: the net with the interconnection added
        """

        df_bus = pd.read_csv(path_bus)

        # find the slack index
        slack_indx = 0
        for ll in range(len(df_bus)):
            # slack_indx = pp.get_element_index(net, "bus", bb.name)
            if df_bus['name'][ll] == 'intercon':
                slack_indx = pp.get_element_index(net, "bus", df_bus['name'][ll])

        pp.create_ext_grid(net, slack_indx, vm_pu=1.0, va_degree=0)

        return net


    def create_trafo(path_trafo):
        """
        defines the transformers

        :param path_trafo:
        :return: the net with the transformers added
        """

        df_trafo = pd.read_csv(path_trafo)

        # for trafo in df_trafo:
        for _, trafo in df_trafo.iterrows():
            hv_bus = pp.get_element_index(net, "bus", trafo.hv_bus)
            lv_bus = pp.get_element_index(net, "bus", trafo.lv_bus)

            pp.create_transformer_from_parameters(net,
                                                  hv_bus,
                                                  lv_bus,
                                                  trafo.sn_mva,
                                                  trafo.vn_hv_kv,
                                                  trafo.vn_lv_kv,
                                                  trafo.vkr_percent,
                                                  trafo.vk_percent,
                                                  trafo.pfe_kw,
                                                  trafo.i0_percent,
                                                  in_service=trafo.in_service)

        return net


    # create empty network
    net = pp.create_empty_network()

    # buses
    net = create_bus(path_bus, path_geodata)

    # lines
    net = create_line(path_line)

    # loads
    net = create_load(path_demand, path_busload, path_bus)

    # gens
    # if rene is False:
        # net = create_generator(path_generation, path_busgen, path_bus)
    # else:
    net = create_generator_rene(path_generation, path_busgen, path_bus)

    # interconnection
    net = create_intercon(path_bus)

    # trafos
    net = create_trafo(path_trafo)

    return net



def get_Plosses(net, prnt=False):
    """
    returns the active power losses

    :param net: full grid
    :param prnt: to print the value
    :return: total active power losses
    """

    Pll = sum(net.res_line['pl_mw'])
    if prnt is True:
        print('The active power losses are: ', Pll, 'MW')

    return Pll


def get_max_loading(net, prnt=False):
    """
    returns the maximum loading in percentage

    :param net: full grid
    :param prnt: to print the value
    :return: maximum percentual loading of the lines
    """

    L_max = max(net.res_line['loading_percent'])
    if prnt is True:
        print('The maximum loading is: ', L_max, '%')

    return L_max


def run_store_timeseries(net, identifier):
    """
    run the timeseries and store the data

    :param net: full grid
    :param identifier: the name to append to the .xlsx file
    :return: nothing, just store in .xlsx
    """

    ow = timeseries.OutputWriter(net, output_path="./Results/Cases/Case_"+identifier, output_file_type=".xlsx")

    ow.log_variable('res_bus', 'vm_pu')
    ow.log_variable('res_line', 'loading_percent')
    ow.log_variable('res_line', 'pl_mw')
    ow.log_variable('line', 'parallel')
    ow.log_variable('load', 'p_mw')  # try to read the load to calculate then the efficiency
    # timeseries.run_timeseries(net)
    timeseries.run_timeseries(net, continue_on_divergence=True)

    # store other data, like state of the lines...
    df_lines_states = pd.DataFrame([net.line['name'], net.line['from_bus'], net.line['to_bus'], net.line['in_service']])
    df_lls = df_lines_states.T
    df_lls.to_excel("./Results/Cases/Case_"+identifier+"/lines_states.xlsx")

    # store diagnostic
    diagn = pp.diagnostic(net, report_style='compact')
    # df_diagn = pd.DataFrame(diagn)
    df_diagn = pd.DataFrame.from_dict(list(diagn))
    df_diagn.to_excel("./Results/Cases/Case_"+identifier+"/diagnostic.xlsx")

    return ()


def perms(n_extra_lines):
    """
    generate the permutations

    :param n_extra_lines: number of added lines to combine
    :return: list of permutations
    """
    # lst = ['True', 'False'] * n_extra_lines
    lst = [True, False] * n_extra_lines
    perms_all = set(itertools.permutations(lst, int(n_extra_lines)))
    perms_all = list(perms_all)

    return perms_all


def run_contingencies_ts(path_bus, path_geodata, path_line, path_demand, path_busload, path_generation, path_busgen, path_trafo, n_extra_lines=0):
    """
    run contingencies by disconnecting lines for now

    :param path: path of the datafiles to create the grid
    :param n_extra_lines: number of added lines to combine, the last ones in the .csv
    :return: store in xlsx files
    """
    perms_extra = perms(n_extra_lines)

    net_ini = initialize_net(path_bus, path_geodata, path_line, path_demand, path_busload, path_generation, path_busgen, path_trafo)

    n_cases = len(perms_extra)
    n_lines = len(net_ini.line)

    # disconnect the initial lines and run the cross cases with connecting the others
    for jj in range(n_lines - n_extra_lines):
        net_2 = pp.pandapowerNet(net_ini)
        net_2.line['in_service'][jj] = False

        # evaluate the state by connecting the extra lines
        for kk in range(n_cases):
            # copy the net
            net_3 = pp.pandapowerNet(net_2)

            # change state of the line (True/False)
            net_3.line['in_service'][n_lines - n_extra_lines:] = perms_extra[kk][:]

            # run timeseries and store
            run_store_timeseries(net_3, str(jj) + '_' + str(kk))
            # run_store_timeseries(net_3, str(jj))
            # run_store_timeseries(net_3, str(hash(str(jj) + '_' + str(kk))))

    # return ()
    return n_lines, n_extra_lines, n_cases


def process_contingencies(n_lines, n_extra_lines, n_cases):
    """
    merge all excels into one but different sheets

    :param n_lines: number of total lines
    :param n_extra_lines: number of added lines
    :param n_cases: resulting total number of cases
    :return: nothing, just store
    """

    dd0 = pd.DataFrame([])

    nnx = n_cases * (n_lines - n_extra_lines)
    f0_vpu = dd0.to_excel("./Results/All_vpu_" + str(nnx) + ".xlsx")
    f0_load = dd0.to_excel("./Results/All_load_" + str(nnx) + ".xlsx")
    f0_pl = dd0.to_excel("./Results/All_pl_" + str(nnx) + ".xlsx")
    f0_diag = dd0.to_excel("./Results/All_diag_" + str(nnx) + ".xlsx")
    f0_line = dd0.to_excel("./Results/All_line_" + str(nnx) + ".xlsx")
    f0_parallel = dd0.to_excel("./Results/All_parallel_" + str(nnx) + ".xlsx") 
    f0_pmw = dd0.to_excel("./Results/All_pmw_" + str(nnx) + ".xlsx")


    w_vpu = pd.ExcelWriter("./Results/All_vpu_" + str(nnx) + ".xlsx")
    w_load = pd.ExcelWriter("./Results/All_load_" + str(nnx) + ".xlsx")
    w_pl = pd.ExcelWriter("./Results/All_pl_" + str(nnx) + ".xlsx")
    w_diag = pd.ExcelWriter("./Results/All_diag_" + str(nnx) + ".xlsx")
    w_line = pd.ExcelWriter("./Results/All_line_" + str(nnx) + ".xlsx")
    w_parallel = pd.ExcelWriter("./Results/All_parallel_" + str(nnx) + ".xlsx")
    w_pmw = pd.ExcelWriter("./Results/All_pmw_" + str(nnx) + ".xlsx")

    for jj in range(n_lines - n_extra_lines):
        for kk in range(n_cases):
            fold_path = "./Results/Cases/Case_" + str(jj) + "_" + str(kk)
            f1_vpu = pd.read_excel(fold_path + "/res_bus/vm_pu.xlsx")
            f1_load = pd.read_excel(fold_path + "/res_line/loading_percent.xlsx")
            f1_pl = pd.read_excel(fold_path + "/res_line/pl_mw.xlsx")
            f1_diag = pd.read_excel(fold_path + "/diagnostic.xlsx")
            f1_line = pd.read_excel(fold_path + "/lines_states.xlsx")
            f1_parallel = pd.read_excel(fold_path + "/line/parallel.xlsx")
            f1_pmw = pd.read_excel(fold_path + "/load/p_mw.xlsx")

            f1_vpu.to_excel(w_vpu, sheet_name=str(jj) + '_' + str(kk))
            f1_load.to_excel(w_load, sheet_name=str(jj) + '_' + str(kk))
            f1_pl.to_excel(w_pl, sheet_name=str(jj) + '_' + str(kk))
            f1_diag.to_excel(w_diag, sheet_name=str(jj) + '_' + str(kk))
            f1_line.to_excel(w_line, sheet_name=str(jj) + '_' + str(kk))
            f1_parallel.to_excel(w_parallel, sheet_name=str(jj) + '_' + str(kk))
            f1_pmw.to_excel(w_pmw, sheet_name=str(jj) + '_' + str(kk))


    w_vpu.save()
    w_load.save()
    w_pl.save()
    w_diag.save()
    w_line.save()
    w_parallel.save()
    w_pmw.save()

    return ()



def find_optimal_config(path_diagN, path_lineN, path_loadN, path_plN, path_vpuN, path_parallelN, path_pmwN, n_lines, n_extra_lines, n_cases, exclude_lines):
    """
    Find the optimal configuration of lines considering convergence, losses, voltages, and costs

    :param path_diagN: path of the diagnostic N cases
    :param path_lineN: path of the line N cases
    :param path_loadN: path of the loading of the lines N cases
    :param path_plN: path of the losses through the lines N cases
    :param path_vpuN: path of the voltages N cases
    :param path_parallelN: path of the number of parallel lines, N cases
    :param path_pmwloadN: path of the MW of load in the buses, N cases

    :return: optimal case
    """

    diag = pd.read_excel(path_diagN, sheet_name=None)
    line = pd.read_excel(path_lineN, sheet_name=None)
    load = pd.read_excel(path_loadN, sheet_name=None)
    pl = pd.read_excel(path_plN, sheet_name=None)
    vpu = pd.read_excel(path_vpuN, sheet_name=None)
    parallel = pd.read_excel(path_parallelN, sheet_name=None)
    pmwload = pd.read_excel(path_pmwN, sheet_name=None)


    on_off_all_lines = []
    vec_config = []
    for name, sheet in diag.items():
        # if len(sheet) == 0 or name[0] not in exclude_lines:  # if no diagnostic errors
        if True or name[0] not in exclude_lines:  # if no diagnostic errors
        # if True:
            on_off_lines = line[name]['in_service']
            on_off_all_lines.append(on_off_lines)
            n_parallel = parallel[name]
            n_parallel_subset = n_parallel.loc[0, 0:].T  # only the first row, all are equal

            # loadings
            loadings = load[name]
            loadings_subset = loadings.loc[0:, 0:]
            conditional_loading = loadings_subset[loadings_subset[:] > 80].isnull().values.all()
            # if True, good, we are below 80%

            # vpu
            vpuss = vpu[name]
            vpuss_subset = vpuss.loc[0:, 0:]
            conditional_vpu1 = vpuss_subset[vpuss_subset[:] > 1.10].isnull().values.all()
            conditional_vpu2 = vpuss_subset[0.01 < vpuss_subset[:]][vpuss_subset[:] < 0.90].isnull().values.all()  # to avoid the 0.0 
            # if both are True, we are good

            # active power losses
            pls = pl[name]
            pls_subset = pls.loc[0:, 0:]
            pmws = pmwload[name]
            pmw_subset = pmws.loc[0:, 0:]

            ok_losses = True
            kk = 0
            while ok_losses is True and kk < 24:  # calculate efficiency at any time
                P_load_all = sum(pmw_subset.loc[kk,:])
                P_loss_all = sum(pls_subset.loc[kk,:])
                P_gen_all = P_load_all + P_loss_all
                eff = P_load_all / P_gen_all
                if eff < 0.98:
                    ok_losses = False
                kk += 1
                # print('Losses: Pload, Ploss, Pgen, eff: ')
                # print(P_load_all, P_loss_all, P_gen_all, eff)

            # if ok_losses is True, we are good

            full_condition = conditional_loading and conditional_vpu1 and conditional_vpu2 and ok_losses
            if full_condition is True:
                print(name)
                vec_config.append(name)

            # print('Condition: load, vpu1, vpu2, loss: ')
            # print(conditional_loading, conditional_vpu1, conditional_vpu2, ok_losses)

    df_configs = pd.DataFrame(vec_config)
    df_configs.to_excel("./Results/OK_configs.xlsx")

    return ()


def select_best(path_configs, path_res_lines, path_ini_lines, n_lines, n_extra_lines, n_cases):
    """
    Find the single optimal configuration, also considering costs

    :param path_configs: path to the OK_configs.xlsx file
    :param path_res_lines: path to the solution for all lines, to read the sheet and states
    :param path_ini_lines: to know the distances
    :return: data for the optimal configuration
    """

    # data costs of the lines
    c_2line = 407521  # euro/km
    c_1line = 288289  # euro/km

    # store costs
    lengths = pd.read_csv(path_ini_lines)['length']
    circuits = pd.read_csv(path_ini_lines)['parallel']
    costs = []
    for ll in range(len(circuits)):
        if circuits.loc[ll] == 1:
            costs.append(c_1line * lengths.loc[ll])
        elif circuits.loc[ll] == 2:
            costs.append(c_2line * lengths.loc[ll])

    costs = np.array(costs)

    # use lines operating state
    res_configs = pd.read_excel(path_configs)
    res_lines = pd.read_excel(path_res_lines, sheet_name=None)
    name_configs_prev = list(res_configs.iloc[:,1])

    # check if the same _yy is in all lines
    configs_all = []
    for xx in name_configs_prev:
        namex = xx.split('_')[1]
        configs_all.append(namex)

    nice_configs = set(configs_all)
    n_count_max = 0
    n_config = '00'

    # look max number of occurrences
    for nn in nice_configs:
        occurr = configs_all.count(nn)
        if occurr > n_count_max:
            n_count_max = occurr

    # store the names of the configurations always valid
    valid_configs = []
    if n_count_max == n_lines - n_extra_lines:
        for nn in nice_configs:
            occurr = configs_all.count(nn)
            if occurr == n_count_max:
                valid_configs.append(nn)




    # only check 0_xx to get the data
    costs_configs = []
    dic_configs = {}
    for name, sheet in res_lines.items():
        nxx = name.split('_')[1]
        if nxx in valid_configs:
            on_off_lines = np.array(res_lines[name]['in_service'])
            c_total = np.dot(on_off_lines, costs)
            dic_configs[nxx] = c_total


    names_final = np.array(list(dic_configs.keys()))
    costs_final = np.array(list(dic_configs.values()))
    # print(dic_configs.keys())
    # print(dic_configs.values())
    print(names_final)
    print(costs_final)

    # final_configs = pd.DataFrame([np.array(dic_configs.keys()), np.array(dic_configs.values())], columns=['configs', 'costs'])
    final_configs = pd.DataFrame([names_final, costs_final])
    # final_configs = pd.DataFrame([np.transpose(names_final), np.transpose(costs_final)])
    print(final_configs)

    final_configs.transpose().to_excel("./Results/OPT_configs.xlsx")

    return ()






if __name__ == "__main__":

    time_start = time.time()


    # ------------- Inputs -------------
    # load paths
    # path_bus = 'Datafiles/phII/bus1.csv'
    # path_geodata = 'Datafiles/phII/geodata1.csv'
    # path_line = 'Datafiles/phII/line2.csv'
    # path_demand = 'Datafiles/phII/demand1.csv'
    # path_busload = 'Datafiles/phII/bus_load1.csv'
    # path_generation = 'Datafiles/phII/generation1.csv'
    # path_busgen = 'Datafiles/phII/bus_gen1.csv'
    # path_trafo = 'Datafiles/phII/trafo1.csv'

    # rising voltage level
    # path_bus = 'Datafiles/phII_380kV/bus1.csv'
    # path_geodata = 'Datafiles/phII_380kV/geodata1.csv'
    # path_line = 'Datafiles/phII_380kV/line2.csv'
    # path_demand = 'Datafiles/phII_380kV/demand1.csv'
    # path_busload = 'Datafiles/phII_380kV/bus_load1.csv'
    # path_generation = 'Datafiles/phII_380kV/generation1.csv'
    # path_busgen = 'Datafiles/phII_380kV/bus_gen1.csv'
    # path_trafo = 'Datafiles/phII_380kV/trafo1.csv'

    # --------------- phase 3 ---------------
    # phase with rene
    # path_bus = 'Datafiles/phIII/bus1.csv'
    # path_geodata = 'Datafiles/phIII/geodata1.csv'
    # path_line = 'Datafiles/phIII/line1reduced.csv'
    # path_demand = 'Datafiles/phIII/demand1.csv'
    # path_busload = 'Datafiles/phIII/bus_load1.csv'
    # path_trafo = 'Datafiles/phIII/trafo1.csv'

    # rene
    # path_generation = 'Datafiles/phIII/generation_all.csv'
    # path_busgen = 'Datafiles/phIII/bus_gen1.csv'

    # no rene
    # path_generation = 'Datafiles/phIII/generation1.csv'
    # path_busgen = 'Datafiles/phIII/bus_gen1_norene.csv'

    # -------------- phase 4 --------------
    path_bus = 'Datafiles/phIV/bus1.csv'
    path_geodata = 'Datafiles/phIV/geodata1.csv'
    path_line = 'Datafiles/phIV/line1reduced.csv'
    path_demand = 'Datafiles/phIV/demand1.csv'
    path_busload = 'Datafiles/phIV/bus_load1.csv'
    path_trafo = 'Datafiles/phIV/trafo1.csv'
    path_generation = 'Datafiles/phIV/generation_all.csv'
    path_busgen = 'Datafiles/phIV/bus_gen1.csv'







    # ------------- Running -------------
    # define net
    # net = initialize_net(path_bus, path_geodata, path_line, path_demand, path_busload, path_generation, path_busgen, path_trafo)
    net = initialize_net(path_bus, path_geodata, path_line, path_demand, path_busload, path_generation, path_busgen, path_trafo, rene=True)

    # run and store timeseries data, for only 1 case
    run_store_timeseries(net, '_00storage')

    # run contingencies
    n_lines, n_extra_lines, n_cases = run_contingencies_ts(path_bus, path_geodata, path_line, path_demand, path_busload, path_generation, path_busgen, path_trafo, n_extra_lines=6)
    process_contingencies(n_lines, n_extra_lines, n_cases)

    # n_lines = 16
    # n_extra_lines = 6
    # n_cases = 64

    print(n_lines, n_extra_lines, n_cases)


    # ------------- Processing -------------
    # store
    nxx = n_cases * (n_lines - n_extra_lines)
    path_diagN = 'Results/All_diag_' + str(nxx) + '.xlsx'
    path_lineN = 'Results/All_line_' + str(nxx) + '.xlsx'
    path_loadN = 'Results/All_load_' + str(nxx) + '.xlsx'
    path_plN = 'Results/All_pl_' + str(nxx) + '.xlsx'
    path_vpuN = 'Results/All_vpu_' + str(nxx) + '.xlsx'
    path_parallelN = 'Results/All_parallel_' + str(nxx) + '.xlsx'
    path_pmwN = 'Results/All_pmw_' + str(nxx) + '.xlsx'

    exclude_lines = [6,7]

    find_optimal_config(path_diagN, path_lineN, path_loadN, path_plN, path_vpuN, path_parallelN, path_pmwN, n_lines, n_extra_lines, n_cases, exclude_lines)

    path_configs = 'Results/OK_configs.xlsx'
    path_lineN = 'Results/All_line_' + str(nxx) + '.xlsx'
    path_line_ini = 'Datafiles/phIV/line1reduced.csv'

    select_best(path_configs, path_lineN, path_line_ini, n_lines, n_extra_lines, n_cases)

    end_time = time.time()
    print(end_time - time_start, 's')



    # ------------- Others -------------

    # run diagnostic
    # pp.diagnostic(net)
    # print(net.res_load)
    # print(net.res_line)
    # print(net.res_gen)

    # plot
    # pp.plotting.simple_plot(net)
    # simple_plot(net)

    # get_Plosses(net, prnt=True)
    # get_max_loading(net, prnt=True)



\end{lstlisting}


\begin{lstlisting}[caption={Code for the calculation of lines}]
import numpy as np

def calc_line(a, b, c, d, e, immax, npar, Rca, Dext, kgg):
    """
    calculate r, x, c, and return also Imax

    :param a: horizontal distance between A1 and C2
    :param b: horizontal distance between B1 and B2
    :param c: horizontal distance between C1 and A2
    :param d: vertical distance between A1 and B1
    :param e: vertical distance between B1 and C1
    :param immax: max current in A
    :param npar: number of parallel lines (1 or 2)
    :param Rca: ac resistance in ohm/km
    :param Dext: external diameter in mm
    :param kg: factor of roughly 0.8
    :return: r, x, c, imax
    """

    def single_line(a, b, immax, Rca, Dext, kgg):
        """
        calculate the R, X, C parameters, also return Imax

        :param a: horizontal distance between A and C
        :param b: vertical distance between A and B
        :param immax: max current in A
        :param Rca: ac resistance in ohm/km
        :param Dext: external diameter in mm
        :param kg: factor of roughly 0.8

        :return: R, X, C, Imax, in the units desired by pandapower
        """

        # cardinal: https://www.elandcables.com/media/38193/acsr-astm-b-aluminium-conductor-steel-reinforced.pdf
        # 54 Al + 7 St, Imax = 888.98 A

        w = 2 * np.pi * 50  # rad / s
        Imax = immax * 1e-3  # kA
        # Stot = 547.3 * 1e-6  # m2, the total section
        # R_ac_75 = 0.07316 * 1e-3  # ohm / m
        # kg = 0.809  # from the slides in a 54 + 7

        R_ac_75 = Rca * 1e-3  # ohm / m, should we correct by temperatures?
        Stot = np.pi * Dext ** 2 / 4 * 1e-6  # m2, the total section
        kg = kgg

        r = np.sqrt(Stot / np.pi)  # considering the total section

        dab = np.sqrt((a / 2) ** 2 + b ** 2)
        dbc = np.sqrt((a / 2) ** 2 + b ** 2)
        dca = a

        GMD = (dab * dbc * dca) ** (1 / 3)
        GMR = kg * r
        RMG = r

        L = 4 * np.pi * 1e-7 / (2 * np.pi) * np.log(GMD / GMR)  # H / m

        C = 2 * np.pi * 1e-9 / (36 * np.pi) / np.log(GMD / RMG)  # F / m

        # in the units pandapower wants
        R_km = R_ac_75 * 1e3  # ohm / km
        X_km = L * w * 1e3  # ohm / km
        C_km = C * 1e9 * 1e3  # nF / km

        return R_km, X_km, C_km, Imax


    def double_line(a, b, c, d, e, immax, Rca, Dext, kgg):
        """
        calculate the R, X, C parameters, also return Imax

        :param a: horizontal distance between A1 and C2
        :param b: horizontal distance between B1 and B2
        :param c: horizontal distance between C1 and A2
        :param d: vertical distance between A1 and B1
        :param e: vertical distance between B1 and C1
        :param immax: max current in A
        :param Rca: ac resistance in ohm/km
        :param Dext: external diameter in mm
        :param kgg: factor of roughly 0.8
        :return: R, X, C, Imax, in the units desired by pandapower
        """

        # cardinal: https://www.elandcables.com/media/38193/acsr-astm-b-aluminium-conductor-steel-reinforced.pdf
        # 54 Al + 7 St, Imax = 888.98 A

        w = 2 * np.pi * 50  # rad / s
        Imax = immax * 1e-3 * 2  # kA, for the full line, x2
        # Stot = 547.3 * 1e-6  # m2, the total section
        # R_ac_75 = 0.07316 * 1e-3  # ohm / m
        # kg = 0.809  # from the slides in a 54 + 7

        R_ac_75 = Rca * 1e-3  # ohm / m, should we correct by temperatures?
        Stot = np.pi * Dext ** 2 / 4 * 1e-6  # m2, the total section
        kg = kgg



        r = np.sqrt(Stot / np.pi)  # considering the total section

        da1b1 = np.sqrt((b / 2 - a / 2) ** 2 + d ** 2)
        da1b2 = np.sqrt((a / 2 + b / 2) ** 2 + d ** 2)
        da2b1 = np.sqrt((c / 2 + b / 2) ** 2 + e ** 2)
        da2b2 = np.sqrt((b / 2 - c / 2) ** 2 + e ** 2)

        db1c1 = np.sqrt((b / 2 - c / 2) ** 2 + e ** 2)
        db1c2 = np.sqrt((b / 2 + a / 2) ** 2 + d ** 2)
        db2c1 = np.sqrt((b / 2 + c / 2) ** 2 + e ** 2)
        db2c2 = np.sqrt((b / 2 - a / 2) ** 2 + d ** 2)

        dc1a1 = np.sqrt((a / 2 - c / 2) ** 2 + (d + e) ** 2)
        dc1a2 = c
        dc2a1 = a
        dc2a2 = np.sqrt((a / 2 - c / 2) ** 2 + (d + e) ** 2)

        dab = (da1b1 * da1b2 * da2b1 * da2b2) ** (1 / 4)
        dbc = (db1c1 * db1c2 * db2c1 * db2c2) ** (1 / 4)
        dca = (dc1a1 * dc1a2 * dc2a1 * dc2a2) ** (1 / 4)

        rp = kg * r

        da1a2 = np.sqrt((a / 2 + c / 2) ** 2 + (d + e) ** 2)
        db1b2 = b
        dc1c2 = np.sqrt((c / 2 + a / 2) ** 2 + (d + e) ** 2)

        drap = np.sqrt(rp * da1a2)
        drbp = np.sqrt(rp * db1b2)
        drcp = np.sqrt(rp * dc1c2)

        dra = np.sqrt(r * da1a2)
        drb = np.sqrt(r * db1b2)
        drc = np.sqrt(r * dc1c2)

        GMD = (dab * dbc * dca) ** (1 / 3)
        GMR = (drap * drbp * drcp) ** (1 / 3)
        RMG = (dra * drb * drc) ** (1 / 3)

        L = 4 * np.pi * 1e-7 / (2 * np.pi) * np.log(GMD / GMR)  # H / m

        C = 2 * np.pi * 1e-9 / (36 * np.pi) / np.log(GMD / RMG)  # F / m

        # in the units pandapower wants
        R_km = R_ac_75 / 2 * 1e3  # ohm / km, like 2 resistances in parallel
        X_km = L * w * 1e3  # ohm / km
        C_km = C * 1e9 * 1e3  # nF / km

        return R_km, X_km, C_km, Imax

    if npar == 1:
        rr, xx, cc, imm = single_line(a, b, immax, Rca, Dext, kgg)
    elif npar == 2:
        rr, xx, cc, imm = double_line(a, b, c, d, e, immax, Rca, Dext, kgg)
    else:
        print('Error: number of parallel lines is not 1 nor 2')

    return rr, xx, cc, imm

# rr, xx, cc, ii = double_line(11, 2, 4, 5, 6, 1000)
# print(rr, xx, cc, ii)

\end{lstlisting}
